\documentclass[francais,a4paper,11pt,reqno]{amsart}

\usepackage[active]{srcltx}
\usepackage{array,delarray}
\usepackage{amsmath,amssymb,amsthm,amsfonts,amsbsy}
\usepackage{latexsym}
\usepackage[utf8]{inputenc}
\usepackage[T1]{fontenc}
\usepackage{babel}

\title{Notes sur le critère de Gelfand}

\author{Florent Hivert}

\date{Juin 2016}

\theoremstyle{plain}
  \newtheorem{THEO}{\bf Théorème}[section]
  \newtheorem{COR}[THEO]{\bf Corollaire}
  \newtheorem{PROP}[THEO]{\bf Proposition}
  \newtheorem{LEMME}[THEO]{\bf Lemme}
  \newtheorem{CONJ}[THEO]{\bf Conjecture}

\theoremstyle{definition}
  \newtheorem{NOTAT}[THEO]{\bf Notation}
  \newtheorem{DEF}[THEO]{\bf Définition}
  \newtheorem{DEFS}[THEO]{\bf Définitions}
  \newtheorem{ALGO}[THEO]{\bf Algorithme}
  \newtheorem{PROB}[THEO]{\bf Problème}

\theoremstyle{remark}
   \newtheorem{EXPL}[THEO]{\bf Exemple}
   \newtheorem{NOTE}[THEO]{\bf Note}

%-*-Latex-*-$Id: notations.tex,v 1.2 2004/10/01 06:39:12 averell Exp $


\newcommand{\NN}{{\mathbb N}}			% Entiers naturels
\newcommand{\ZZ}{{\mathbb Z}}			% Anneau des entiers relatifs
\newcommand{\FF}{{\mathbb F}}			% Corps fini
\newcommand{\QQ}{{\mathbb Q}}			% Corps des rationnels
\newcommand{\CC}{{\mathbb C}}			% Corps des complexes
\newcommand{\KK}{{\mathbb K}}			% Corps de base

\newcommand{\pairing}[2]{\left\langle#1|#2\right\rangle} % Crochet de dualite
\newcommand{\partie}{\operatorname{\mathcal P}}	% Ensemble des parties
\newcommand{\card}{\operatorname{\#}}		% Cardinal d'un ensemble
\newcommand{\Endom}{\operatorname{End}}		% Endomorphisme
\newcommand{\Lin}{\operatorname{\mathcal L}}		% Endomorphisme


% Quelques abbreviations
\def\FCT.{fonction}
\def\POL.{polyn�me}
\def\SY.{sym�trique}
\def\QS.{quasi-sym�trique}
\def\NC.{non commutative}
\newcommand{\SF}{\FCT.s \SY.s}
\newcommand{\QSF}{\FCT.s \QS.s}
\newcommand{\NCSF}{\FCT.s \SY.s \NC.s}

\newcommand{\question}{\Large\bf}
\newcommand{\ie}{{\it i.e.\ }}
\newcommand{\cf}{{\it cf.\ }}
\newcommand{\eg}{{\it e.g.\ }}
\newcommand{\resp}{{\it resp.\ }}


% Ensembles divers et varies
\newcommand{\X}{{X}}				% Alphabet
\newcommand{\SG}{{\mathfrak S}}			% Groupe symetrique
\newcommand{\idSG}{\operatorname{Id}}
\newcommand{\Heck}{{H}}				% Algebre de Hecke
\newcommand{\AffHeck}{\widetilde{H}}		% Algebre de Hecke
\newcommand{\sym}{\mbox{Sym}}			% Fonctions symetriques
\newcommand{\QSym}{\mbox{\rm QSym}}		% Fonctions quasi-symetriques
\newcommand{\Sym}{\mbox{\bf Sym}}		% Fonctions non commutative
\newcommand{\FSym}{\mbox{\bf FSym}}
\newcommand{\PBT}{\mbox{\bf PBT}}
\newcommand{\FQSym}{\mbox{\bf FQSym}}
\newcommand{\WQSym}{\mbox{\bf WQSym}}
\newcommand{\MQSym}{\mbox{\bf MQSym}}		% commutative
\newcommand{\pol}[1]{[#1]}			% algebre des polynomes en X
\newcommand{\free}[1]{\langle#1\rangle}		% algebre libre sur x
\newcommand{\Prim}{\operatorname{Prim}}		% primitive elements
\newcommand{\Lie}{\operatorname{\mathcal L}}	% algebre de Lie libre sur x
\newcommand{\Shuffle}{\operatorname{Shuf}}	% algebre de Lie libre sur x
\newcommand{\Hilb}{\operatorname{Hilb}}	        % Serie de Hilbert
\newcommand{\mat}[1]{\{#1\}}			% algebre des multimots. 

% Operateur divers et varies
						% quasi-symetrique
\newcommand{\tr}{\operatorname{Tr}}		% Trace
\newcommand{\caract}{\operatorname{ch}}		% Caracteristique
\newcommand{\carac}{\chi}			% Caractere
\newcommand{\ind}{\operatorname{Ind}}		% Rep. induite
\newcommand{\res}{\operatorname{Res}}		% Rep. restreinte
\newcommand{\rien}{{\!}}			% Vecteur vide pour egalite
						% entre operateurs
\newcommand{\compl}{\backslash}			% Complementaire
\newcommand{\isom}{\simeq}			% isomorphe.
\newcommand{\inti}[2]{[\![#1,#2]\!]} 

% operateurs de Sn et Hn(q)
\newcommand{\id}{\operatorname{Id}}		% Identite
\newcommand{\sgn}{\operatorname{Sgn}}		% Signe
\newcommand{\maxperm}{\operatorname{\omega}}	% Permutation maximale
\newcommand{\sign}{\operatorname{\epsilon}}	% Signe d'une permutation
\newcommand{\perm}{\operatorname{\sigma}}	% Permutation
\newcommand{\Qperm}{\operatorname{\boldsymbol{\sigma}}}
\newcommand{\heck}{\operatorname{\mathit{T}}}	% generateur de Hn(q)
\newcommand{\Qheck}{\operatorname{\mathbfit{T}}}
\newcommand{\heckb}%
   {\operatorname{\,\overline{\mathit{\!T}}}}	% generateur de Hn(q)
\newcommand{\Qheckb}%
   {\operatorname{\overline{\mathbfit{T}}}}	% generateur de Hn(q)
\newcommand{\Dun}{\operatorname{\mathit{Y}}}
\newcommand{\QDun}{\operatorname{\mathbfit{Y}}}
\newcommand{\Rot}{\operatorname{\Omega}}	%
\newcommand{\QRot}{\operatorname{\boldsymbol\Omega}}
\newcommand{\ddd}{\operatorname{\partial}}
\newcommand{\dd}{\operatorname{\pi}}		% difference divisee pi_i
\newcommand{\Qdd}{\operatorname{\boldsymbol\pi}}	
\newcommand{\ddb}{\operatorname{\overline\pi}}	% difference divisee pibar_i
\newcommand{\Qddb}{\operatorname{\overline{\boldsymbol\pi}}}
\newcommand{\ddT}{\operatorname{\mathit{T}}}	% difference divisee T_i
\newcommand{\QddT}{\operatorname{\mathbfit{T}}}
\newcommand{\ddC}{\operatorname{\Box}}		% difference divisee Carre_i
%\newcommand{\QddC}{\operatorname{\pmb{\boldsymbol{\Box}}}}
\newcommand{\QddC}{\operatorname{\pmb{\Box}}}
\newcommand{\ddN}{\operatorname{\nabla}}	% difference divisee Nabla_i
\newcommand{\QddN}{\operatorname{\boldsymbol{\nabla}}}		
\newcommand{\Tab}{\operatorname{Tab}}

\newcommand{\charact}{\operatorname{ch}}        % Caracteristique
\newcommand{\charac}{\chi}                      % Caractere

% definition de U(Gln)
\newcommand{\Univ}{\operatorname{U}}	        % Algebre envelopante
\newcommand{\gl}{{gl}}				% Algebre de lie
\newcommand{\GL}{{GL}}				% Groupe lineaire
\newcommand{\Ugl}{\Univ_0(\gl_N)}		% U_0(GL_N)
\newcommand{\cartan}{\mathfrak{h}}
\newcommand{\Cartan}{{H}}			% Groupe lineaire
\newcommand{\borel}{\mathfrak{b}_{+}}
\newcommand{\Borel}{{B^{+}}}
\newcommand{\eGen}{\operatorname{\mathit{e}}}	% generateur de U(gl_n)
\newcommand{\fGen}{\operatorname{\mathit{f}}}	% generateur de U(gl_n)
\newcommand{\kGen}{\operatorname{\mathit{k}}}	% generateur de U(gl_n)
\newcommand{\copro}{\Delta}			% Co-produit
\newcommand{\coun}{\epsilon}			% Co-unite
\newcommand{\tensor}{\otimes}			% Produit tensoriel
\newcommand{\diag}{\operatorname{Diag}}

\newcommand{\F}{\mathbf{F}}

\newcommand{\Space}{{V}}                        % Espace vectoriel
\newcommand{\Dim}{{N}}                          % Dimension de V
\newcommand{\base}{\xi}              		% Vecteur de la base
\newcommand{\baset}{\boldsymbol{\xi}}		%  ... en gras
\newcommand{\baseend}{{E}}                      % base des endomorphismes
\newcommand{\funrep}{\rho}                      % representation
\newcommand{\irred}{\operatorname{\mathbf{D}}}	% module irreductible
\newcommand{\qcrys}{\Gamma}			% graph quasi-cristallin
\newcommand{\weyl}{{W}}				% groupe de Weyl

% Compositions
\newcommand{\lon}{\operatorname{\ell}}          % Longeur
\newcommand{\des}{\operatorname{Des}}           % Ensemble des descentes
\newcommand{\sumcomp}{\triangleright}           % Somme de deux compositions
\newcommand{\sumvect}[1]{|#1|}			% Somme totale d'un vecteur
\newcommand{\catcomp}{\cdot}			% Concatenation "
\newcommand{\cset}{\operatorname{C}}            % ensemble de descente
\newcommand{\bre}{\operatorname{\#}}            % Breaking composition
\newcommand{\fin}{{\succeq}}                    % Ordre de raffinement
\newcommand{\sfin}{{\succ}}			%  ... Strict
\newcommand{\conj}[1]{#1\,\tilde{}\,}		% Conjugaison
\newcommand{\mir}[1]{\overline#1}		% Image mirroir
\newcommand{\maj}{\operatorname{Maj}}           % Indice majeur
\newcommand{\partof}{\vdash}                    % Partition de
\newcommand{\compof}{\vDash}                    % Composisition de
\newcommand{\domin}{\trianglerighteq}

\newcommand{\comp}{K}
\newcommand{\compe}{k}
\newcommand{\kip}{\comp=(\compe_1,\ldots,\compe_p)}
\newcommand{\pseudo}[1]{[#1]}
\newcommand{\suite}[2]{#1_1,\ldots,#1_{#2}}
\newcommand{\parsum}[2]{#1_1+\cdots+#1_{#2}}
\newcommand{\spec}[1]{_{/_{\scriptstyle#1}}}      % Specialisation

% Mots
\newcommand{\nat}{\operatorname{Nat}}
\newcommand{\motu}{\mathbf{u}}
\newcommand{\motun}{\motu=u_1\ldots u_n}
\newcommand{\motv}{\mathbf{v}}
\newcommand{\motvn}{\mott=v_1\ldots v_n}
\newcommand{\eval}{\operatorname{Eval}}
\newcommand{\evalt}{\trans\eval}
\newcommand{\QR}{\operatorname{QR}}

\newcommand{\Sh}{\shuffle}
\newcommand{\ShBar}{\shufflebar}
\newcommand{\std}{\operatorname{Std}}

% Alg�bre de Hopf
\newcommand{\hopf}{\mahtcal{H}}
\newcommand{\conc}{\mathrm{conc}}
\newcommand{\produ}{\times}
\newcommand{\Produ}{\mu}
\newcommand{\cprod}{\delta}
\newcommand{\Cprod}{\mathbf{c}}
\newcommand{\dtb}{\underline\cprod}
\newcommand{\cunit}{\epsilon}
\newcommand{\Cunit}{\mathbf{e}}
\newcommand{\apode}{\alpha}
\newcommand{\Apode}{\mathbf{a}}
\newcommand{\p}{\operatorname{\!\cdot\!}}
\newcommand{\vide}{e}
\newcommand{\convol}{\operatorname{Convol}}

\newcommand{\Rel}[1][{}]{\,\mathcal{R}_{#1}\,}
\newcommand{\TODO}[1]{\textbf{\large To Do : #1}}
\newcommand{\qandq}{\quad\text{et}\quad}
\newcommand{\Peak}{\operatorname{\sf Peak}}
\newcommand{\Ind}{\operatorname{Ind}}
\newcommand{\Res}{\operatorname{Res}}
\newcommand{\Hom}{\operatorname{Hom}}
\newcommand{\Center}{\operatorname{Z}}



% texte de definition des groupe symetrique et algebre de hecke
\newcommand{\forallini}{\text{pour $1\leq i \leq n-1$,}}
\newcommand{\foralliji}{\text{pour $|i-j|>1$,}}
\newcommand{\forallind}{\text{pour $1\leq i \leq n-2$.}}
\newcommand{\defhecke}[4]{%
  \begin{alignat}{2}\label{#4}
   #1_i^2&=#2                                \qquad&&\forallini\notag\\
   #1_i#1_j&=#1_j#1_i                        \qquad&&\foralliji\\
   #1_i#1_{i+1}#1_i &=#1_{i+1}#1_i#1_{i+1}#3 \qquad&&\forallind\notag
   \end{alignat}%
}

\newenvironment{DefCases}{\begin{array}\{{c>{\ $}l<{$}}.}{\end{array}}
\newenvironment{EquCases}{\begin{array}\{{r@{}l>{\qquad$}l<{$}}.}{\end{array}}

%%% Local Variables: 
%%% mode: latex
%%% TeX-master: "GenQSym"
%%% End: 

\newcommand{\act}{\cdot}
\newcommand{\simple}{S}

\newcommand\eqcom[1]{\mathrel{\overset{\text{#1}}{=}}}

%%%%%%%%%%%%%%%%%%%%%%%%%%%%%%%%%%%%%%%%%%%%%%%%%%%%%%%%%%%%%%%%%%%%%%%%%%%%%%
\begin{document}
\maketitle

\section{Restrictions et modules sur le commutant}

Soit $B$ une algèbre sur $\CC$ et $A$ une sous-algèbre. Soit $U$ un $A$-module
et $V$ un $B$-module. On considère l'espace vectoriel
\begin{equation}
H := \Hom_A(U, \res^B_A V)\,.
\end{equation}
On note $Z := Z(B, A)$ la sous algèbre de $B$ des éléments qui commutent avec
$A$.
\begin{LEMME}
  $H$ est un $Z$-module pour l'action
  \begin{equation}
    z\act h := u\mapsto z \act h(u).
  \end{equation}
\end{LEMME}
\begin{proof}
  On montre que $z\act h\in H$:
  \begin{multline}
    (z\act h)(a\act u) \eqcom{def} z\act h(a\act u)
    \eqcom{morph} z\act (a \act h(u)) \\
    \eqcom{act}(za)\act h(u)
    \eqcom{def $Z$}(az)\act h(u)
    \eqcom{act}a\act(z\act h(u))
    \eqcom{def}a\act(z\act h)(u)\,.
  \end{multline}
  De plus,
  \begin{equation}
    (z\act (z'\act h))(u) = (zz'\act h(u)) = (zz'\act h)(u)\,.
  \end{equation}
  $H$ est donc bien un $Z$-module.
\end{proof}
Le but ce cette section est de montrer que si $U$ et $V$ sont simples (sur
resp. $A$ et $B$), alors $H$ est simple sur $Z$.


\begin{LEMME}
  On note $\Lin(U, V)$ l'ensemble des applications $\CC$-linéaire de $U$ dans
  $V$. C'est un $(B\tensor_k A)$ bimodule ($B$-mod-$A$) avec les actions
  suivantes:
  \begin{equation}
    b\act f := u\mapsto b\act f(u)
    \qandq
    f\act a := u\mapsto f(a\act u)\,.
  \end{equation}
\end{LEMME}
\begin{proof}
  C'est un $B$-mod:
  \begin{equation}
    b\act (b'\act f) = u\mapsto b \act (b'\act(f(u)))
    = u\mapsto b b'\act(f(u)) = bb'\act f\,.
  \end{equation}
  C'est un mod-$A$:
  \begin{multline}
    (f\act a) \act a'
    \eqcom{def}(u\mapsto f(a\act u)) \act a'
    \eqcom{def}u\mapsto f(a\act (a' \act u))\\
    \eqcom{act}u\mapsto f(aa'\act u)
    \eqcom{def}f\act aa'\,.\qedhere
  \end{multline}
  Les deux structures de module commutent:
  \begin{equation}
    b\act (f\act a) = u\mapsto b\act f(a\act u) = (b\act f)\act a\,.
  \end{equation}
\end{proof}
\begin{NOTE}
  On peux aussi écrire $\Lin(U, V)$ comme $V\tensor_k U^*$ où $U^*$ est le
  module dual de $U$, c'est-à-dire l'espace des formes linéaire sur $U$ avec
  l'action
  \begin{equation}
  (\phi\act a) := u\mapsto \phi(a\act u)\,.
\end{equation}
\end{NOTE}
\bigskip

L'espace vectoriel sous-jacent à $B$ est aussi un $(B\tensor_k A)$-bimodule
par les multiplication à gauches et à droite. Il y a donc un sens à considèrer
l'espace vectoriel des morphismes de $(B\tensor_k A)$-bimodules de $B$ dans
$\Lin(U,V)$.
\begin{equation}
  L := \hom_{B\tensor_K A}(B, \Lin(U, V))\,.
\end{equation}
\begin{LEMME}
  $L$ est un $Z$-module pour l'action
  \begin{equation}
    z\act l := b\mapsto l(bz)
  \end{equation}
\end{LEMME}
\begin{proof}
  Montrons que $z\act l\in L$\TODO{}

  C'est une action:
  \begin{multline}
    z\act(z'\act l)
    \eqcom{def}z\act(b \mapsto l(b z'))
    \eqcom{def}b \mapsto l((b z) z') \\
    =b \mapsto l(b (zz')) \eqcom{def}(zz') \act l\,.\qedhere
  \end{multline}
\end{proof}

L'idée est de transformer le $\Hom_A$ de la définition de $H$ en un $\Lin$ ou
ce qui revient au même, un produit tensoriel sur $A$ en un produit tensoriel
sur $k$.
\begin{PROP}
  Pour $h\in H$ et $l\in L$, on pose
  \begin{equation}
    \hat{h} := b \mapsto (u \mapsto b\act h(u))
    \qandq
    \tilde{l} := u \mapsto l(1)(u)\,.
  \end{equation}
  Alors, les applications $h\mapsto\hat{h}$ et $l\mapsto\tilde{l}$ sont deux
  isomorphismes réciproques des $Z$-modules $H$ et $L$. En particulier, ils
  sont isomorphes.
\end{PROP}
\begin{proof}
  \begin{itemize}
  \item[$\bullet$] Si $h\in H$ alors $\hat{h}\in L$.
    \begin{equation}
      \hat h(b'\act b\act a) = \hat h(b'ba) = u\mapsto b'ba \act h(u)
      = b'\act b\act h(a\act u) = b' \act \hat{h} \act a.
    \end{equation}
  \item[$\bullet$] $\hat{.}$ est un $Z$-morphisme.
    \begin{align}
      \widehat{z\act h} = \widehat{u\mapsto z\act h(u)}
      &\eqcom{def}b\mapsto(u\mapsto b\act(z\act h(u)))\\
      &\eqcom{act}b\mapsto(u\mapsto bz\act h(u))\\
      &\eqcom{def}z\act(b\mapsto(u\mapsto b\act h(u)))=z\act\hat{h}\,.
    \end{align}
  \item[$\bullet$] Si $l\in L$ alors $\tilde{l}\in H$.
    \begin{multline}
      \tilde{l}(a\act u) = l(1)(a\act u)
      \eqcom{morph A}l(1\act a)(u)\\
      = l(a\act 1)(u)
      \eqcom{morph B}a\act l(1)(u) = (a\act \tilde{l})(u)\,.
    \end{multline}
  \item[$\bullet$] $\tilde{.}$ est un $Z$-morphisme.
    \begin{align}
      \widetilde{z\act l}&=u\mapsto (z\act l)(1)(u)\\
      &\eqcom{def}u\mapsto l(1z)(u)\\
      &=u\mapsto l(z\act 1)(u)\\
      &\eqcom{morph}z\act (u\mapsto l(1)(u))= z\act \tilde{l}
    \end{align}
  \item[$\bullet$] Pour tout $h\in H$, on a l'égalité $\tilde{\hat{h}} = h$.
    \begin{equation}
      \tilde{\hat{h}} = \widetilde{b \mapsto (u \mapsto b\act h(u))}
      = u \mapsto 1\act h(u) = h
    \end{equation}
  \item[$\bullet$] Pour tout $l\in L$, on a l'égalité $\hat{\tilde{l}} = l$.
    \begin{align}
      \hat{\tilde{l}} &= \widehat{u \mapsto l(1)(u)}\\
      &\eqcom{def}b \mapsto (u \mapsto b\act l(1)(u)) \\
      &\eqcom{morph}b \mapsto u \mapsto l(b)(u) = l\,.\qedhere
    \end{align}
  \end{itemize}
\end{proof}

\begin{PROP}
  Si $A$ et $B$ sont semi-simple et si $U$ et $V$ sont des modules simples,
  alors le $Z$-module $L$ est simple. Par conséquent $H$ est aussi simple.
\end{PROP}
\begin{proof}
  Soit $f,g\in L$ deux éléments non nuls. Ce sont donc deux homorphismes de
  $(B, A)$-bimodule de $B$ dans $\Lin(U,V)$. Le problème est de trouver un
  $z\in Z$ que $z\act g = f$.

  Les modules $U$ et $V$ étant simple sur $A$ et $B$, le bi-module $\Lin(U,V)$
  est simple sur $B\tensor_k A$. Comme $f$ et $g$ sont non nuls, par le lemme
  de Schur il sont surjectifs. Ils définissent donc deux isomorphismes:
  \begin{equation}
    B / \ker{f} \isom \Lin(U,V)
    \qandq
    B / \ker{g} \isom \Lin(U,V)\,.
  \end{equation}
  Par semi-simplicité de $B\tensor_k A$, on peut trouver des supplémentaires
  stables $M_f$ et $M_g$ à $\ker{f}$ et $\ker{g}$ qui sont donc également
  isomorphe. On peut donc construire un isomorphisme $\phi$ de
  $(B,A)$-bimodule qui envoie $\ker f$ sur $\ker g$ et $M_f$ sur $M_g$.
  Ainsi:
  \begin{itemize}
  \item si $b\in\ker f$, alors $\phi(b)\in\ker g$ et donc
    $g\circ \phi(b) = f(b)$.
  \item si $b\in M_f$, alors $g\circ \phi(b) = f(b)$ par les isomorphismes ci
    dessus.
  \end{itemize}
  Ainsi $g\circ \phi = f$. Mais comme $\phi$ commute avec l'action de $b$, il
  existe un $z\in B$ tel que pour tout $b\in B$, on a $\phi(b) = bz$. Pour
  tout $b\in b$, on a donc $g\circ \phi(b) = g(bz) = (z\act g)(b)$. Il reste à
  montrer que $z\in Z$. Soit donc $a\in A$. On vérifie que:
  \begin{equation}
    az = \phi(a) = \phi(1)\act a = z\act a = za\,.\qedhere
  \end{equation}
\end{proof}

\begin{THEO}[Critère de Gelfand]
  $B$ est sans multiplicité sur $A$ si et seulement si $Z(B, A)$ est
  commutatif.
\end{THEO}
\begin{proof}
  Supposons $B$ sans multiplicité sur $A$. \TODO{}
  
  Inversement, supposons que $Z(B, A)$ est commutatif. On a montré que pour
  tout $U$, $V$ simples, l'espace $\Hom_A(U, \res^B_A V)$ est un $Z$-module
  simple, il est donc de dimension $1$.
\end{proof}


\section{Centre et application du critère de Gelfand}

\subsection{Le centre d'une algèbre de groupe}

Rappel sur l'algèbre de groupe $\CC G$.
\begin{LEMME}
  Si $c\in\Center(\CC G)$ alors $c$ agit par un scalaire sur toute les
  représentations.
\end{LEMME}
\begin{proof}
  C'est une conséquence du lemme de Schur.
\end{proof}

\begin{LEMME}
  L'application $f\mapsto \sum_g f(g^{-1})g$ est un isomorphisme linéaire
  des fonctions centrales dans $\Center(\CC G)$.
\end{LEMME}

\begin{PROP}[Caractère et centre]
  Soit $(\rho, V)$ une représentations et $f$ une fonction
  centrale. Alors
  \begin{equation}
    \tr_V\left(\sum_g f(g^{-1})g\right) = |G|\langle f\mid\chi_\rho\rangle\,.
  \end{equation}
\end{PROP}
En particulier les caractères irréductibles sont à un scalaire près les
idempotents centraux:
\begin{PROP}[Caractère irréductible et idempotents centraux]
  Soit $(\rho, V)$ et $(\eta, W)$ deux représentations irréductibles.  On pose
  $e_\rho := \frac{\chi_\rho(1)}{|G|} z(\rho)$.
  \begin{itemize}
  \item si $\rho$ et $\eta$ ne sont pas isomorphes alors $e_\rho$ agit par $0$
    sur $W$.
  \item si $\rho$ et $\eta$ sont isomorphes alors $e_\rho$ agit par $1$ sur $W$.
  \end{itemize}
  Par conséquent $e_\rho$ est un idempotent tel que pour toute représentation
  $(\eta,W)$, l'espace $\eta(e_\rho)(W)$ est la composante isotypique associée
  à $\rho$ de $W$.
\end{PROP}
\begin{NOTE}
  Interprétation à la lumière de Wederburn:
  \begin{equation}
    \CC G \isom \bigoplus_{\rho\in\hat{G}} M_{\dim \rho}(\CC)
  \end{equation}
  Les idempotents centraux sont les identités des $M_{\dim \rho}(\CC)$.
\end{NOTE}
Why do we care ?
\begin{COR}
  Les composantes isotypiques sont canoniques et ne dépende pas de la
  décomposition choisie.
\end{COR}

\subsection{Graphe de branchement et base de Gelfand-Tzetlin}
Voir section 1 article Vershik-Okounkov.

\begin{DEF}[Graphe de branchement (diagramme de Bratelli) d'une chaine de
  groupe]
  \begin{equation}
    G_0=\{1\} \subset G_1 \subset G_2 \subset G_3 \subset \dots
  \end{equation}
  Multi-graphe gradué:
  \begin{itemize}
  \item les sommets de degré $n$ sont les $\lambda\in\hat{G_n}$.
  \item pour $\lambda\in\hat{G_n}$ et $\mu\in\hat{G_{n+1}}$ le nombre d'arètes
    entre $\lambda$ et $\mu$ est égale à $\dim\Hom(V_\lambda, \Res^{n+1}_nV_\mu)$.
  \end{itemize}
\end{DEF}
\begin{PROP}
  Pour toute irrep $(\lambda, V)$, alors $\dim(V)$ est le nombre de chemin de $0$
  à $\lambda$ dans le graphe de branchement.
\end{PROP}
Si la chaîne est sans multiplicité, le multigraphe est un graphe. Comme la
décomposition en isotypique est canonique, on obtient une décomposition
canonique de $V_\lambda$ en droites. À des scalaires près on obtient une base
canonique de $V_\lambda$ qui est appelée la base de Gelfand-Tzetlin.
\begin{DEF}
  Algèbres de Gelfand-Tzetlin $GZ_n$ sous algèbre $\CC G_n$ engendrée par
  les $\Center(\CC G_i)$ pour $i\leq n$.
\end{DEF}
C'est une sous-algèbre commutative de $\CC G_n$ (en fait elle est maximale).

D'où l'intérêt du critère de Gelfand.

\subsection{Le cas des groupes symétriques}

La chaînes des groupes symétriques est sans multiplicités:
\begin{THEO}
  L'algèbre $\Center_{n-1,n} := \Center(\CC \SG_n, \CC \SG_{n-1})$ est commutative.
\end{THEO}
\begin{LEMME}
  Tout élément de $\sigma\in\SG_n$ est conjugué à son inverse $\sigma^{-1}$
  par un élément de $\SG_{n-1}$.
\end{LEMME}
Pour $x=\sum_g x_g g\in\CC G$ notons $x^* := \sum_g x_g g^{-1}$. C'est un
anti-isomorphisme d'algèbre.
\begin{LEMME}
  Pour $c\in\Center_{n-1,n}$, on a $c^* = c$.
\end{LEMME}

\begin{proof}[Démonstration du théorème]
  Si $a,b\in\Center_{n-1,n}$ alors
  \begin{equation}
    (a-ib)(a+ib) = (a^*-ib^*)(a^*+ib^*) =  ((a-ib)(a+ib))^* = (a+ib)(a-ib)
  \end{equation}
  On en déduit que $ab=ba$.
\end{proof}

Ce qu'il faut maintenant faire: Décrire $\Center_{n-1,n}$ et l'algèbre de
Gelfand-Tzetlin des groupes symétrique.

\subsection{Éléments de Jucis-Murphy}

La somme de toute les transpositions $\sum_{i<j\leq n} (i,j)$ est un élément du
centre de $\CC\SG_n$. Par conséquent, l'élément de Jucis-Murphy défini par
\begin{equation}
  X_n := \sum_{i<n} (i,n) = \sum_{i<j\leq n} (i,j) - \sum_{i<j\leq n-1} (i,j)
\end{equation}
est un élément de $\Center_{n-1,n}$ et de $GZ(n)$. En particulier, les
$(X_i)_{i\leq n}$ commutent.
\begin{THEO}
  \begin{equation}
    \Center(\CC\SG_n) \subset \langle \Center(\CC\SG_{n-1}), X_n\rangle
  \end{equation}
\end{THEO}
\begin{proof}
  A détailler dans une autre séance. Relier au fait que les fonctions
  puissances engendrent les polynômes symétriques.
\end{proof}
\begin{COR}
  L'algèbre de Gelfand-Tzetlin est engendré par les éléments de Jucis-Murphy:
  \begin{equation}
    GZ_n = \langle X_1, X_2, \dots, X_n \rangle\,.
  \end{equation}
\end{COR}
\begin{THEO}
  $\Center(\CC \SG_n, \CC \SG_{n-1}) =
  \langle \Center(\CC\SG_{n-1}), X_n\rangle$.
\end{THEO}
\end{document}
%%% Local Variables:
%%% mode: latex
%%% TeX-master: t
%%% End: 
