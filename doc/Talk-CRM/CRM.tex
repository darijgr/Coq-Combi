\documentclass[compress,11pt]{beamer}

\usetheme{TALK}
\usepackage[vcentermath]{genyoungtabtikz}
\usepackage{minted}
\usemintedstyle{emacs}

\usepackage{genyoungtabtikz}

\newminted{coq}{
frame=lines,
framesep=2mm,
fontsize=\scriptsize,
mathescape=true
}
\usepackage{commath}

% INFO DOCUMENT - TITRE, AUTEUR, INSTITUTION
\title{\bf\LARGE A formal proof of the \\
Littlewood-Richardson rule\\[5mm]}
\author{Florent Hivert}
\institute[LRI]{
  LRI / Université Paris Sud 11 / CNRS}
\date[September 2018]{September 2018}

\newcommand{\XX}{{\mathbb X}}

\newcommand{\free}[1]{\left\langle#1\right\rangle}
\newcommand{\N}{{\mathbb N}}
\newcommand{\C}{{\mathbb C}}
\newcommand{\Q}{{\mathbb Q}}
\newcommand{\SG}{{\mathfrak S}}
\newcommand{\std}{\operatorname{Std}}

\newcommand{\sym}{\mathrm{sym}}
\newcommand{\NCSF}{\mathbf{NCSF}}
\newcommand{\QSym}{\mathrm{QSym}}
\newcommand{\FSym}{\mathbf{FSym}}

\newcommand{\partof}{\vdash}                    % Partition de
\newcommand{\compof}{\vDash}                    % Composisition de

\newcommand{\qandq}{\text{\quad et\quad}}

\newcommand{\alphX}{{\mathbb X}}
%%%%%%%%%%%%%%%%%%%%%
\newcommand{\red}[1]{{\color{red} #1}}
\newcommand{\grn}[1]{{\color{green} #1}}
\newcommand{\blu}[1]{{\color{blue} #1}}
%------------------------------------------------------------------------------
\begin{document}

% PAGE D'ACCUEIL
\frame{\titlepage}

\begin{frame}{Outline}
  \tableofcontents
\end{frame}

\section{Motivation : certified proof in combinatorics}
%%%%%%%%%%%%%%%%%%%%%%%%%%%%%%%%%%%%%%%%%%%%%%%%%%%%%%%%
\begin{frame}{Why formalize things on computers}

  \begin{center}
    \textbf{\LARGE Writing correct programs is hard:}
  \end{center}
\medskip

\begin{itemize}
\item The human mind is focused on the big picture;
\item Hard to take track of all the trivial / particular cases.
\end{itemize}
\bigskip\pause

Some excerpts of my contribution to \texttt{Sagemath}:
\begin{itemize}
\item determinant / rank / invertibility of $0\times0$ and $1\times1$ matrices
\item empty set and its permutation
\item empty partition / composition / parking function / tableau \dots
\item the $0$ and $1$ species
\item \dots
\end{itemize}
\bigskip\pause

\centering{\textbf{\huge What about proofs ?}}
\end{frame}

\begin{frame}{Are our proofs always correct ?}

Donald Knuth:
\begin{quote}\large
  Beware of bugs in the above code; I have only proved it correct, not tried it.
\end{quote}
\bigskip\pause

Often in combinatorics, and particularly in \textbf{bijective} combinatorics,
proofs are algorithms, together with justifications that they meet their
specifications...
\end{frame}

\begin{frame}{Are our proofs always correct ?}

  The Littlewood-Richardson rule:

  \begin{itemize}
  \item stated (1934) by D. E. Littlewood and A. R. Richardson, \red{wrong
      proof, wrong example}.
  \item Robinson (1938), wrong completed proof.
  \item more wrong published proofs...
  \item first correct proof: Schützenberger (1977).
  \item nowadays: dozens of different proofs\dots
  \end{itemize}

  \begin{quotation}\small
    \textbf{Wikipedia}: The Littlewood–Richardson rule is \textbf{notorious
      for the number of errors} that appeared prior to its complete, published
    proof. Several published attempts to prove it are incomplete, and it is
    particularly difficult to avoid errors when doing hand calculations with
    it: even the original example in D. E. Littlewood and A. R. Richardson
    (1934) contains an error.
 \end{quotation}
\end{frame}


\begin{frame}{The case of the Littlewood-Richardson rule ?}

  A footnote in Macdonald's book:
  \medskip

  \begin{quotation}
    \noindent
    Gordon James reports that he was once told that:

    ``The Littlewood-Richardson rule helped to \grn{get men on the
      moon}, but it was \red{not proved until after they had got
      there}. The first part of this story might be an exaggeration.''
  \end{quotation}
  \bigskip

  This sentence appears in James, G. D. (1987) \emph{The representation theory
    of the symmetric groups}.
\end{frame}

\begin{frame}{The case of the Littlewood-Richardson rule ?}
More quotation of James:

\begin{quotation}\small
  It seems that for a long time \textbf{the entire body of experts in the
    field was convinced} by these proofs; at any rate it was not until 1976
  that McConnell pointed out \textbf{a subtle ambiguity} in part of the
  construction underlying the argument.

  [...]

  How was it possible for an \red{incorrect proof} of such a central result in
  the theory of $S_n$ to have been \red{accepted for close to forty years}?
  The level of rigor customary among mathematicians when a combinatorial
  argument is required, is (\grn{probably quite rightly}) of the
  \grn{nonpedantic hand-waving} kind; perhaps one lesson to be drawn is that
  a \textbf{higher degree of care} will be needed in dealing with such
  combinatorial complexities as occur in the present level of development of
  Young's approach.
\end{quotation}
\end{frame}

\begin{frame}[fragile]
  \begin{problem}
    Suppose that, back in 1977, they had had our current proof assistant
    technology. Would it have been \textbf{feasible} to check Schützenberger
    proof ? If so, \textbf{how long} would it have taken ?
  \end{problem}
  \bigskip\pause

  \begin{Theorem}[Constructive answer !]
    Yes ! Less than 5 month and two weeks !
  \end{Theorem}

\small
\begin{verbatim}
commit f990146b8c6e062fe025740a08f888deb9481c2d
Date:   Thu Jul 24 17:46:58 2014 +0200
    Schensted's algorithm.

commit 2418282695455261e5459b33d3e8f979d57c3bdb
Date:   Sun Jan 4 15:31:16 2015 +0100
    DONE the proof of the Littlewood_Richardson rule !!!!
\end{verbatim}
\end{frame}

\section{A short introduction to formal proof in Coq/Mathcomp}
%%%%%%%%%%%%%%%%%%%%%%%%%%%%%%%%%%%%%%%%%%%%%%%%%%%%%%%%%%%%%%%%%
\begin{frame}{History of Coq and Mathcomp}

  \begin{itemize}
  \item 1985 -- T.~Coquand :  \emph{Calculus of constructions}
  \item 1989 -- T.~Coquand, G.~Huet: creation of \texttt{Coq}
  \item 2004 -- G.~Gonthier, B.~Werner : \emph{4 color theorem} in Coq

    Along their way \texttt{Ssreflect} ``small scale reflection''.

  \item 2006 -- 2018 Mathematical component: a library of formalized mathematics.
    \begin{itemize}
    \item basic data structures, algebra, group an representation theory;
    \item the infrastructure for the machine checked proofs of:
    \end{itemize}
  \item 2012 -- Coq checked proof of Feit-Thomson's theorem:

    Every finite group of odd order is solvable.
  \end{itemize}
\end{frame}

\begin{frame}{Formal (mechanized) proofs}

  \begin{block}{Aim}
    Write a proof that is checked by computer all the way down to the
    logical foundation.
  \end{block}
  \bigskip\pause

  \textbf{Proof assistant / interactive theorem prover} : \\ A kind of Integrated
  Development Environment (IDE) which helps writing such proofs by constantly
  checking the coherence and keeping track of missing parts.  \bigskip

  Note: Proof \grn{assistant} $\neq$ \red{automated} theorem prover
\end{frame}


\begin{frame}{What is needed to build a proof assistant ?}

Three ingredients:
\begin{enumerate}
\item A way to \textbf{\red{store algorithms}} that allows for\\
  \textbf{manipulating} them \textbf{and reasoning} about them;
\bigskip\pause

\item A way to \textbf{\red{store proofs}} that allows for\\
  \textbf{manipulating} them \textbf{and reasoning} about them;
\bigskip\pause

\item A way to \textbf{mechanically check} everything.
\end{enumerate}
\end{frame}

\begin{frame}{Proofs as programs (Curry-Howard)}

Suppose that
\begin{itemize}
\item we have \textbf{data} encoding a proof $a$ and two statements $A$ and
  $B$
\item the system is able to make so-called \textbf{judgments}: \\
  to verify that $a$ is a correct proof of $A$ (written as $a : A$)
\end{itemize}
\pause\medskip

Then, the statement $A \to B$ means that each time we have a proof of $A$, we
can construct a proof of $B$.
\pause\bigskip

\begin{block}{Curry-Howard correspondence in a nutshell}
  The idea is ``simply'' to \textbf{encode a proof of $A \to B$ by a function}\\
  (= a program) which takes a proof of $A$ and returns a proof of $B$.
  \bigskip\pause

  Similarly, a proof of $\forall x, P(x)$, is encoded as a function which
  takes $x$ and returns a proof of $P(x)$.
\end{block}


\end{frame}

\begin{frame}{Type theory based proof assistants}

  Proof assistant = a system that:
    \pause\medskip

  \begin{itemize}
  \item \textbf{manipulates} (stores, executes, \dots) functions
    \hfill ($\Lambda$-calculus)
    \medskip

  \item \textbf{checks judgments} such as $a : A$\hfill(typed $\Lambda$-calculus)
  \end{itemize}
  \pause\medskip

  To make it more usable, we need also
  \begin{itemize}
  \item \textbf{building blocks} for custom data structures: \textbf{records,
      unions} \\
    \hfill (Calculus of Inductive Construction $\approx$ Galina)

  \item \textbf{helpers} for writing proof/programs incrementally \\
    \hfill (tactic language).
  \end{itemize}
\end{frame}

\begin{frame}

  You only need to remember:
  \begin{block}{Summary}
    \begin{itemize}
    \item proof, statement, data, programs, etc are all the same first class
      manipulated objects called \textbf{terms}
    \item some terms are allowed (from the logic or by their definition) to
      appear on the right of the judgment symbol ``$:$''. They are called
      \textbf{types}. They encode usual \textbf{data types} as well as
      \textbf{statement}
    \item every term has a type
    \end{itemize}

  \end{block}
  \pause

  \Huge\bf

  Enough for the theory\dots
  \bigskip

  \hspace{2cm}Time for a demo\dots
\end{frame}

\begin{frame}[fragile]{Boolean reflection}

Two ways to deal with statements:
\begin{itemize}
\item \textbf{inductive formulas} (\textit{i.e.} data structure storing a
  proof): \texttt{and, or, exist\dots}:

  $\Rightarrow$ good for reasoning, deducing, implication chaining\dots

  \bigskip

\item \textbf{decision procedure} (\textit{i.e.} function returning a
  boolean):

  $\Rightarrow$ good for combinatorial analysis, automatically taking care of
  trivial cases\dots
\end{itemize}

\begin{block}{Boolean reflection}
  Going back and forth between the two ways:
\end{block}
  \begin{coqcode}
    reflect (maxn m n = m) (m >= n).
    reflect (exists2 x : T, x \in s & a x) (has a s)
    reflect (filter s = s) (all s)
    reflect (forall x, x \in s -> a x) (all a s).
    reflect (exists2 i, i < size s & nth x0 s i = x) (x \in s).
  \end{coqcode}
\end{frame}


\section{The Littlewood-Richardson rule}

\begin{frame}[fragile]{Integer Partitions}

  Partition: $\lambda := (\lambda_0\geq\lambda_1\geq\dots\geq\lambda_l > 0)$.\\
  $|\lambda| := \lambda_0+\lambda_1+\dots+\lambda_l \qandq
  \ell(\lambda) := l\,. $

  Ferrers diagram of a partitions : $(5,3,2,2) \leftrightarrow \yngs(0.5, 2,2,3,5)$

  {\tiny\hfill
    \href{file:html/Combi.Combi.partition.html#is_part}{\texttt{is\_part}}
  \vspace{-2mm}}
\begin{coqcode}
  Fixpoint is_part sh := (* Predicate *)
    if sh is sh0 :: sh'
    then (sh0 >= head 1 sh') && (is_part sh')
    else true.

  Lemma is_partP sh : reflect (* Boolean reflection lemma *)
    (last 1 sh != 0 /\ forall i, (nth 0 sh i) >= (nth 0 sh i.+1))
    (is_part sh).

  Lemma is_part_ijP sh : reflect (* Boolean reflection lemma *)
    (last 1 sh != 0 /\ forall i j, i <= j -> (nth 0 sh i) >= nth 0 sh j)
    (is_part sh).

  Lemma is_part_sortedE sh : (is_part sh) = (sorted geq sh) && (0 \notin sh).
\end{coqcode}

\end{frame}

\begin{frame}{Symmetric Polynomials}

  $n$-variables : $\XX_n := \{x_0, x_1, \dots x_{n-1}\}$.

  Polynomials in $\XX$ : $\C[\XX] = \C[x_0, x_1, \dots, x_{n-1}]$; ex: $3x_0^3x_2
  + 5 x_1x_2^4$.

  \begin{DEFN}[Symmetric polynomial]
    A polynomial is \emph{symmetric} if it is invariant under any permutation of the
    variables: for all $\sigma\in\SG_n$,
    \[P(x_0, x_1, \dots, x_{n-1}) = 
    P(x_{\sigma(0)}, x_{\sigma(1)}, \dots, x_{\sigma({n-1})})\]
  \end{DEFN}

  \[P(a,b,c) = a^2b + a^2c + b^2c + ab^2 + ac^2 + bc^2\]
  \[Q(a,b,c) = 5abc + 3a^2bc + 3ab^2c + 3abc^2\]

\end{frame}

\begin{frame}{Schur symmetric polynomials (Jacobi)}

  \begin{DEFN}[Schur symmetric polynomial]
    \small Partition $\lambda :=
    (\lambda_0\geq\lambda_1\geq\dots\geq\lambda_{l-1})$ with $l\leq
    n$; set $\lambda_i:=0$ for $i\geq l$.
    \[
    s_{\lambda} = 
    \frac{\sum_{\sigma\in\SG_n} \operatorname{sign}(\sigma)
      \XX_n^{\sigma(\lambda+\rho)}}%
    {\prod_{0\leq i<j<n} (x_j - x_i)}
    = \frac{\left|
       \begin{smallmatrix}
         x_1^{\lambda_{n-1}+0}  & x_2^{\lambda_{n-1}+0}   & \dots  & x_n^{\lambda_{n-1}+0}  \\
         x_1^{\lambda_{n-2}+1}  & x_2^{\lambda_{n-2}+1}   & \dots  & x_n^{\lambda_{n-2}+1}  \\
         \vdots & \vdots & \ddots & \vdots \\
         x_1^{\lambda_1+n-2}  & x_2^{\lambda_1+n-2}   & \dots  & x_n^{\lambda_1+n-2}  \\
         x_1^{\lambda_0+n-1}  & x_2^{\lambda_0+n-1}   & \dots  & x_n^{\lambda_0+n-1}  \\
      \end{smallmatrix}
      \right|
    }{\left|
       \begin{smallmatrix}
         1      & 1      & \dots  & 1     \\
         x_1    & x_2    & \dots  & x_n    \\
         x_1^2  & x_2^2   & \dots  & x_n^2  \\
         \vdots & \vdots & \ddots & \vdots \\
         x_1^{n-1}  & x_2^{n-1}   & \dots  & x_n^{n-1}  \\
      \end{smallmatrix}
      \right|
    }
    \]
  \end{DEFN}
  \[s_{(2,1)}(a,b,c) = a^2b + ab^2 + a^2c + 2abc + b^2c + ac^2 + bc^2\]
\end{frame}


\begin{frame}{Littlewood-Richardson coefficients}

  \begin{PROP}
    The family $(s_\lambda(\XX_n))_{\ell(\lambda) \leq n}$ is a (linear) basis of the
    ring of symmetric polynomials on $\XX_n$.
  \end{PROP}

  \begin{DEFN}[Littlewood-Richardson coefficients]
    Coefficients $c_{\lambda,\mu}^\nu$ of the expansion of the product:
    \[
    s_\lambda s_\mu = \sum_{\nu} c_{\lambda,\mu}^\nu s_\nu\,.
    \]
  \end{DEFN}
  Fact: $s_\lambda(\XX_{n-1}, x_n := 0) =
  s_\lambda(\XX_{n-1}) \text{ if } \ell(\lambda) < n\text{ else } 0$.

  Consequence: $c_{\lambda,\mu}^\nu$ are independant of the number of variables.
\end{frame}

\Yboxdim{10pt}
\begin{frame}{Young Tableau}

  \begin{DEFN}
    \begin{itemize}
    \item Filling of a partition shape;
    \item non decreasing along the rows;
    \item strictly increasing along the columns.
      \medskip
    \item row reading
    \end{itemize}
  \end{DEFN}
  \[\scriptsize
  \young(dde,bcccd,aaabbde) = ddebcccdaaabbde
  \qquad\qquad
  \young(5,269,13478)=526913478\]

\end{frame}

\begin{frame}[fragile]{Young Tableau: formal definition}

  {\tiny\hfill
    \href{file:html/Combi.Combi.tableau.html#dominate}{\texttt{dominate}}
  \vspace{-2mm}}
\begin{coqcode}
  Variable (T : ordType) (Z : T). (* Ordered type Order <A *)

  Definition dominate (u v : seq T) :=
    (size u <= size v &&
     (all (fun i => nth Z u i >A nth Z v i) (iota 0 (size u))).

  Lemma dominateP u v :
    reflect (size u <= size v /\
             forall i, i < size u -> nth Z u i >A nth Z v i)
            (dominate u v).
\end{coqcode}
  {\tiny\hfill
    \href{file:html/Combi.Combi.tableau.html#is_tableau}{\texttt{is\_tableau}}
  \vspace{-2mm}}
\begin{coqcode}
  Fixpoint is_tableau (t : seq (seq T)) :=
    if t is t0 :: t' then
      [&& (t0 != [::]), sorted t0,
        dominate (head [::] t') t0 & is_tableau t']
    else true.

  Definition to_word t := flatten (rev t).
\end{coqcode}
\end{frame}

\begin{frame}{Combinatorial definition of Schur functions}

  \begin{DEFN}
    \[s_\lambda(\XX) = \sum_\text{$T$ tableaux of shape $\lambda$} \XX^T\]
    where $\XX^T$ is the product of the elements of $T$.
  \end{DEFN}
  \[s_{(2,1)}(a,b,c) = a^2b\,+\,ab^2\,+\,a^2c\ +\quad 2abc\quad +\
  b^2c\,+\,ac^2\,+\,bc^2\]
  \[s_{(2,1)}(a,b,c) = {  \Yboxdim{8pt}\scriptsize
    \young(b,aa) + \young(b,ab) + \young(c,aa)
  + \young(b,ac) + \young(c,ab) + \young(c,bb) + \young(c,ac) +
  \young(c,bc)}\]
  \pause\medskip

  Note: I'll prove the equivalence of the two definitions as a
  \emph{consequence of a particular case of the LR-rule (Pieri rule)} by
  relating it with recursively unfolding determinants.
\end{frame}

\begin{frame}[fragile]

  {\tiny\hfill
    \href{file:html/Combi.Combi.tableau.html#tabsh}{\texttt{tabsh}}
  \vspace{-2mm}}
\begin{coqcode}
Variable n : nat.
Variable R : comRingType.  (* Commutative ring *)

(* 'I_n         : integer in 0,1,...,n-1                             *)
(* 'P_d         : partition of the integer d                         *)
(* {mpoly R[n]} : the ring of polynomial over the commutative ring R *)
(*                in n variables (P.-Y. Strub)                       *)

Definition is_tab_of_shape sh :=
  [ pred t :> seq (seq 'I_n.+1) | (is_tableau t) && (shape t == sh) ].

Structure tabsh sh := TabSh {tabshval; _ : is_tab_of_shape sh tabshval}.
[...]
Canonical tabsh_finType sh := [...] (* Finite type *)
\end{coqcode}
  {\tiny\hfill
    \href{file:html/Combi.MPoly.Schur_mpoly.html#Schur}{\texttt{Schur}}
  \vspace{-2mm}}
\begin{coqcode}
Definition Schur d (sh : 'P_d) : {mpoly R[n]} :=
  \sum_(t : tabsh n sh) \prod_(i <- to_word t) 'X_i.
\end{coqcode}
\end{frame}

\begin{frame}{Yamanouchi Words}

  $\abs{w}_x$ : number of occurrence of $x$ in $w$.

  \begin{DEFN}
    Sequence $w_0,\dots,w_{l-1}$ of integers such that for all $k, i$,
    \[ \abs{w_i,\dots,w_{l-1}}_k \geq \abs{w_i,\dots,w_{l-1}}_{k+1} \]

    Equivalently $(\abs{w}_i)_{i\leq\max(w)}$ is a partition and $w_1,\dots,w_{l-1}$ is
    also Yamanouchi.
  \end{DEFN}

  \[ (), 0, 00, 10, 000, 100, 010, 210, \]
  \[ 0000, 1010, 1100, 0010, 0100, 1000, 0210, 2010, 2100, 3210 \]
\end{frame}

\begin{frame}{The LR Rule at last !}

  \begin{THEO}[Littlewood-Richardson rule]
    $c_{\lambda, \mu}^{\nu}$ is the number of (skew) tableaux of shape the
    difference $\nu/\lambda$, whose row reading is a Yamanouchi word of
    evaluation $\mu$.
  \end{THEO}
% ...00 ...00 ...00
% ...1  ...1  ...1 
% .00   .01   .02  
% 12    02    01   
  \[
  C_{331,421}^{5432} = 3
  \qquad \Yboxdim{7pt}\tiny
  \gyoung(12,:;00,:::;1,:::;00)\qquad
  \gyoung(02,:;01,:::;1,:::;00)\qquad
  \gyoung(01,:;02,:::;1,:::;00)
  \]
% ...00 ...00 ...00 ...00 ...00 ...00 ...00 ...00 ...00 ...00 ...00 ...00 ...00 ...00 ...00
% ...1  ...1  ...1  ...1  ...1  ...1  ...1  ...1  ...1  ...1  ...1  ...1  ...1  ...1  ...1
% .00   .00   .00   .01   .00   .01   .02   .02   .12   .02   .02   .01   .01   .12   .12
% 01    02    11    01    12    02    01    13    03    03    11    12    22    02    23
  \[
  C_{431,4321}^{7542} = 4
  \qquad \Yboxdim{7pt}\tiny
  \gyoung(23,:;112,:::;01,::::;000)\quad
  \gyoung(23,:;012,:::;11,::::;000)\quad
  \gyoung(13,:;022,:::;11,::::;000)\quad
  \gyoung(03,:;122,:::;11,::::;000)\quad
  \]
  \[
  C_{4321,431}^{7542} = 4
  \qquad \Yboxdim{7pt}\tiny
  \gyoung(:;2,::;11,:::;01,::::;000)\quad
  \gyoung(:;1,::;12,:::;01,::::;000)\quad
  \gyoung(:;1,::;02,:::;11,::::;000)\quad
  \gyoung(:;0,::;12,:::;11,::::;000)\quad
  \]

\end{frame}

\begin{frame}[fragile]{The formal statement}

  {\tiny\hfill
    \href{file:html/Combi.LRrule.therule.html#is_skew_reshape_tableau}%
    {\texttt{definition of LR-yam tableaux}}
  \vspace{-2mm}}
\begin{coqcode}
(** yameval P = type of Yamanouchi word of evaluation P *)
Lemma is_skew_reshape_tableauP (w : seq nat) :
  size w = sumn (P / P1) ->
  reflect
    (exists tab, [/\ is_skew_tableau P1 tab,
                     shape tab = P / P1 & to_word tab = w])
    (is_skew_reshape_tableau P P1 w).

Definition LRyam_set :=
  [set y : yameval P2 | is_skew_reshape_tableau P P1 y].
Definition LRyam_coeff := #|LRyam_set|.
\end{coqcode}
Then
  {\tiny\hfill
    \href{file:html/Combi.LRrule.therule.html#LRyam_coeffP}%
    {the Littlewood-Richardson rule}
  \vspace{-2mm}}
\begin{coqcode}
Theorem LRyam_coeffP :
  Schur P1 * Schur P2 =
  \sum_(P : 'P_(d1 + d2) | included P1 P) Schur P *+ LRyam_coeff P.
\end{coqcode} 

\end{frame}

\section{Some hard points of the formal proof}

\begin{frame}[fragile]{Getting definition right}

There is not a single ``good'' definition:
\bigskip

\begin{itemize}
\item Lots of different equivalent ways. (eg:
\href{file:html/Combi.Combi.partition.html#is_part}{partitions},
\href{file:html/Combi.Combi.tableau.html#is_tableau}{tableaux})
\bigskip\pause

\item Even more difficult for algorithms (
\href{file:html/Combi.Combi.std.html#std_rec}{standardization},
\href{file:html/Combi.LRrule.shuffle.html#shuffle_from_rec}{shuffle}):
  \medskip

  Constraints:
  \begin{itemize}
  \item \textbf{Pure functional} programming: \\
    no variable, no mutable data structure
    \bigskip
  \item All function must be \textbf{total} (e.g. \texttt{nth} but \texttt{option}) \\
    \bigskip
  \item Only trivially \textbf{terminating programs} are allowed:\\
    Only recursive call on subterms are allowed.

  \end{itemize}
\end{itemize}
\end{frame}

\begin{frame}[fragile]{Choices in constructive mathematics}

Sometime you have a choice to make where \textbf{any choice will do}:
\bigskip\pause

Example: contructing a \textbf{conjugating permutation} between two
permutations with same cycle type: {\tiny\hfill
  \href{file:html/Combi.SymGroup.cycletype.html#conj_permP}%
  {\texttt{Conjugacy classes of $S_n$}} \vspace{-2mm}}
\begin{coqcode}
(* (s ^ c)%g == s conjugated by c *)
Theorem conj_permP s t :
  reflect (exists c, t = (s ^ c)%g) (cycle_type s == cycle_type t).
\end{coqcode}
\begin{itemize}
\item currently, you have to write a \textbf{precise program} to make the
  choice and prove that it works
\item the proof is usually easy because any choice will do
\item but writing the program making the choice can be harder than expected.
\end{itemize}


\end{frame}

\begin{frame}[fragile]{Equality / set theory}

  \textbf{\Large Coq is based on CIC $\neq$ set theory.}\bigskip

  \begin{itemize}
  \item Constructive logic (not that much a problem in combinatorics)\\
    Excluded middle \verb|P \/ ~ P| is not provable (but can be added as an
    axiom). \bigskip

  \item SSReflect deals smoothly with objects with decidable equality 
    \medskip

    This forbids generating series ! \\

    but see C. Cohen, B. Djalal
    \emph{Formalization of a Newton Series Representation of Polynomials}
    \bigskip

  \item The equality in type theory is ``stronger'' that in set theory

    No proof of functional extensionality:\\
    \[\texttt{(forall x, f x = f y) -> f = g}\]
  \end{itemize}
\end{frame}

\begin{frame}[fragile]{Greene Theorem}

  Disjoint support increasing subsequences:
  \[
  \begin{array}{c@{}c@{}c@{}c@{}c@{}c@{}c@{}c@{}c@{}c@{}c@{}c@{}c}
    \grn{a}&\grn{b}&\red{a}&\grn{b}&\grn{c}&\red a&\red b&\red b&\blu{a}&\red d&\blu{b}&a&\blu{b}
  \end{array}
  \]
  $RS(w)$: Robinson-Schensted tableau of $w$:
  \pause
  \begin{THEO}
    For any word $w$, and integer $k$
    \begin{itemize}
    \item The sum of the length of the $k$-th first row of $RS(w)$ is the maximum
      sum of the length of $k$ disjoint support increasing subsequences of $w$;
    \end{itemize}
  \end{THEO}
\end{frame}

\begin{frame}[fragile]{Equality on dependent type nightmare}

Subsequences of a word $w$ encoded by subsets of the indices of the letter
of $w$: \verb|{set 'I_(size w)}|. But, when
\begin{itemize}
\item \verb|x := u ++ [:: a; b] ++ v|
\item \verb|y := u ++ [:: b; a] ++ v|
\end{itemize}
\texttt{x} and \texttt{y} are two different words ! \\
\verb|{set 'I_(size x)}| and \verb|{set 'I_(size y)}|: different types \\
\begin{block}{Equality on dependent type}
  One can only write \texttt{u = v} if \texttt{u} and \texttt{v} are of the
  \textbf{same type}.
\end{block}
\end{frame}

\begin{frame}[fragile]{Cast between dependent type nightmare}

Here is a solution:
\begin{itemize}
\item Prove that \verb|Hcast : size x = size y|
\item Then \verb|cast_ord Hcast : 'I_(size w) -> 'I_(size y)|
\item Then define:
  {\tiny\hfill
    \href{file:html/Combi.SSRcomplements.ordcast.html#cast_set}{\texttt{cast\_set}}
  \vspace{-2mm}}
\begin{coqcode}
(*  f @: S == image of S by the function f *)
Definition cast_set n m (H : n = m) : {set 'I_n} -> {set 'I_m} :=
  [fun s : {set 'I_n} => (cast_ord H) @: s].
\end{coqcode}
  {\tiny\hfill
    \href{file:html/Combi.LRrule.Greene_inv.html#NoSetContainingBoth.swap_set}{\texttt{swap\_set}}
  \vspace{-2mm}}
\begin{coqcode}
Definition swap (i : 'I_(size x)) : 'I_(size x) :=
  if i == pos0 then pos1 else if i == pos1 then pos0 else i.
Definition swap_setX :=
  [fun S : {set 'I_(size x)} => swap @: S : {set 'I_(size x)}].
Definition swap_set : {set 'I_(size x)} -> {set 'I_(size y)} :=
  (fun S : {set 'I_(size x)} => 
       [set cast_ord Hcast x | x in S]) \o swap_setX.
\end{coqcode}

\end{itemize}
\end{frame}

\section{Should you try ?}

\begin{frame}{Should you try {\tiny(or is this a big waste of time)} ? My two cents}

  \begin{itemize}
  \item I'm pretty convinced (I'm not the only one: Voevodsky, Hales) that
    in the future (how far ?), formal math (not Coq/CIC) will becomes very
    important (as is computation today).  \bigskip\pause

  \item However, currently, experts are \textbf{not satisfied
      with the foundation} (equality\dots). \bigskip\pause

  \item This was \textbf{much easier} that I first expected !
    \bigskip\pause

  \item Boolean reflection : very good job \textbf{dealing with trivial cases}
  \end{itemize}
\end{frame}

\begin{frame}{Should you try {\tiny(or is this a big waste of time)} ? My two cents}

  \begin{itemize}
  \item Lots of time spent on \textbf{reusable basic stuff}\\
    (tableau / partition / rewriting systems / symmetric fncts)\dots
    \bigskip

  \item Some
    other~\href{file:html/index.html}{results}:\medskip
    \begin{itemize}
    \item The hook-length formulas: 3 weeks (joint work w. C. Paulin)
      \medskip
    \item Cycle decomposition: 2 months (T. Benjamin, undergrad)
      \medskip
    \item Basic theory of symmetric functions: 3 months
      \medskip
    \item Character theory of the symmetric groups: 1 month
    \end{itemize}
    \bigskip\pause

  \item This is transforming math into a video-game \medskip

    \hfill\grn{\bf Fun if you like it, but very addictive !}
  \end{itemize}
\end{frame}

\begin{frame}{Want to have a closer look ?}
  \Large

  The code:\\
  \begin{itemize}
  \item \url{https://github.com/hivert/Coq-Combi}
  \end{itemize}
  \bigskip  \bigskip

  The documentation:\\
  \begin{itemize}
  \item \url{http://hivert.github.io/Coq-Combi}
  \end{itemize}
\end{frame}
\end{document}
% "latex -synctex=1 -shell-escape"
%%% Local Variables: LaTeX-command
%%% mode: latex
%%% TeX-master: t
%%% End: 
