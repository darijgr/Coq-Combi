\documentclass[12pt,a4paper]{article}

\usepackage[utf8]{inputenc}
\usepackage[T1]{fontenc}
\usepackage[top=30pt,bottom=30pt,left=48pt,right=46pt]{geometry}

\usepackage{amsmath,amssymb,amsfonts,amsthm}
\usepackage[vcentermath]{genyoungtabtikz}
% \YFrench % use french convention for tableaux.

\usepackage{minted}
\usemintedstyle{emacs}
%\usemintedstyle{colorful}
%\usemintedstyle{borland}
%\usemintedstyle{autumn}

\newminted{coq}{
frame=lines,
framesep=2mm,
mathescape=true
}
\usepackage{commath}

\newcommand{\Coq}{\texttt{Coq}}
\newcommand{\SSR}{\texttt{SSReflect}}
\newcommand{\LR}{Littlewood-Richardson\ }


% INFO DOCUMENT - TITRE, AUTEUR, INSTITUTION
\title{\bf\LARGE A formal proof of \\
\LR rule\\[5mm]}
\author{Florent Hivert}
%\institute[LRI]{
%  LRI / Université Paris Sud 11 / CNRS / INRIA}
\date{Mai 2015}

\newcommand{\free}[1]{\left\langle#1\right\rangle}
\newcommand{\N}{{\mathbb N}}
\newcommand{\C}{{\mathbb C}}
\newcommand{\Q}{{\mathbb Q}}
\newcommand{\SG}{{\mathfrak S}}
\newcommand{\std}{\operatorname{Std}}

\newcommand{\sym}{\mathrm{sym}}
\newcommand{\NCSF}{\mathbf{NCSF}}
\newcommand{\QSym}{\mathrm{QSym}}
\newcommand{\FSym}{\mathbf{FSym}}

\newcommand{\partof}{\vdash}                    % Partition de
\newcommand{\compof}{\vDash}                    % Composisition de

\newcommand{\qandq}{\text{\quad et\quad}}

\newcommand{\red}[1]{{\color{red} #1}}
\newcommand{\grn}[1]{{\color{green} #1}}
\newcommand{\blu}[1]{{\color{blue} #1}}

\newcommand{\alphX}{{\mathbb X}}



\newtheorem{THEO}{Theorem}
\newtheorem{PROP}{Proposition}
\newtheorem{LEMMA}{Lemma}
\newtheorem{CORO}{Corollary}
\newtheorem{PROBLEM}{Problem}
\newtheorem{REMARK}{Remark}
\newtheorem{NOTE}{Note}

% \theoremstyle{definition}
\newtheorem{DEFN}{Definition}
\newtheorem{DEFNs}{Definitions}
\newtheorem{ALGO}{Algorithm}

%------------------------------------------------------------------------------
\begin{document}

\maketitle

\abstract{We present a formalized proof of the \LR rule using \Coq{} and
  \SSR{}. The \LR coefficients are defined as the coefficient of the
  expansion of the product of two Schur function (a particular basis of the
  ring of symmetric functions). Those coefficient are nonnegative
  integers. The \LR rule describe allows to compute these coefficients as
  the number of filling of a specific shape with integer satisfying some
  comparison conditions.}

\tableofcontents

\section{Introduction}

We start by a general presentation of the problem. The goal is to give a
reader an idea of the field of algebraic combinatorics and the \LR problem. In
this first part we stay rather sketchy: to avoid to much repetition, precise
definitions will be given later together with their Coq formalization.

\subsection{Symmetric function and \LR coefficients}
The ring \emph{symmetric functions} is defined as a limit as $n$ goes to the
infinity of the ring of symmetric polynomial in $n$ indeterminates. This ring
serves as universal structure in which relations between symmetric polynomials
can be expressed in a way independent of the number $n$ of indeterminates (but
its elements are neither polynomials nor functions). Among other things, this
ring plays an important role in the representation theory of the symmetric
group or the general linear group~\cite{MacDo}. It also plays a role in
various geometric problems~\cite{Horn,Grassman}.

In these various context, the most important ingredient is a particular linear
basis $(s_\lambda)_\lambda$ whose elements are called the \emph{Schur
  functions}. Recall that linear basis of symmetric functions are indexed by
\emph{integer partition}, that is non increasing sequences of positive
integers. As a linear basis of an algebra, the product of two Schur functions
can be expressed as a linear combination of Schur functions:
\begin{equation}
  s_\lambda s_\mu = \sum_{\nu} c_{\lambda, \mu}^{\nu}\ s_\nu\,.
\end{equation}
For example,
\begin{multline}
  s_{(2,1)} * s_{(3,2,2)} = s_{(3,2,2,2,1)} + s_{(3,3,2,1,1)} + s_{(3,3,2,2)} +
  s_{(3,3,3,1)}
  + s_{(4,2,2,1,1)} + \\
  s_{(4,2,2,2)} + 2s_{(4,3,2,1)} + s_{(4,3,3)} + s_{(4,4,2)} + s_{(5,2,2,1)} +
  s_{(5,3,2)}
\end{multline}
The coefficients $c_{\lambda, \mu}^{\nu}$ of the decomposition are called the
\LR coefficients and are nonnegative integer. The \LR rule describe them as
the number of certain combinatorial configurations called \LR tableaux. For
example, from the previous expansion one can read that $c_{((2,1),
  (3,2,2)}^{(4,3,2,1)} = 2$ which correspond to the two following
configurations:
\begin{equation}
  \gyoung(2,12,:;01,::;00)\qquad\qquad
  \gyoung(2,02,:;11,::;00)
\end{equation}
The precise definition of the configuration is rather intricate, crossing
several types of constraints on the filling of a diagram of box by numbers. It
makes the rule difficult to state, to use and even more to prove: According to
Wikipedia~\cite{WikiLR}:
\begin{quotation}
  The \LR rule is notorious for the number of errors that appeared prior to
  its complete, published proof. Several published attempts to prove it are
  incomplete, and it is particularly difficult to avoid errors when doing hand
  calculations with it: even the original example in D.~E.~Littlewood and
  A.~R.~Richardson (1934) contains an error.
\end{quotation}

This rule was first stated in 1934 by D.~E.~Littlewood and
A.~R.~Richardson~\cite{LR}. However, their proof was wrong: they only proved
it in a very particular case. They also made a mistake in their example. In
1938, Robinson~\cite{Robinson} attempted to complete the proof, but there was
still a mistake. One has to wait until 1977 to get the first correct proof due
to Schützenberger. This proof has numerous combinatorial ingredient, and as it
was written, many combinatorialists though that this proof was ``somewhat
gappy''. The present work follows more or less this original proof showing
that there where actually no crucial gaps. After this first proof, one can
find dozens of thesis and paper about simplifying the
argument~\cite{Zelevinsky81,Macdonald95,Gasharov98,DHT01,VanLeeuwen01,Stembridge02}
and the combinatorial study of these coefficients it is still an active
research topic~\cite{qAnalogs}.  \bigskip

\subsection{A computational point of view on \LR coefficients}

From a computational point of view, such a rule give a good way to compute
those numbers. Indeed it was proved in 2006 by H.~Narayanan that the
computation of the \LR coefficients is
$\#P$-complete~\cite{Narayanan06}. Recall the $\#P$ is the complexity class
counting problem (i.e. with an answer in $\N$) analog of the complexity class
$NP$ for decision problem (i.e. with a boolean answer). More formally, $\#P$
is the class of function problems of the form "compute $f(x)$", where $f$ is
the number of accepting paths of a nondeterministic Turing machine running in
polynomial time. This roughly means that we shouldn't expect to have a better
algorithm to compute those number in general than enumerating the solution of
a combinatorial problem such as \LR tableaux. 

Note that there are other combinatorial model for them such as Knutson and Tao
Honeycomb~\cite{KnutsonTao}. This model appear in particular in ...
\LR coefficient
also appear in Mulmuley's Geometric complexity theory, a strategy to prove
that $P\neq NP$ using invariant theory.

\begin{quotation}
  We point out that the remarkable Knutson and Tao Saturation Theorem and
  polynomial time algorithms for LP have together an important and immediate
  consequence in Geometric Complexity Theory. The problem of deciding
  positivity of Littlewood-Richardson coefficients for GLn(C) belongs to
  P. Furthermore, the algorithm is strongly polynomial.  The main goal of this
  article is to explain the significance of this result in the context of
  Geometric Complexity Theory. Furthermore, it is also conjectured that an
*  analogous result holds for arbitrary symmetrizable Kac-Moody algebras.
\end{quotation}

\subsection{Algebraic point of view on \LR coefficients}

From the algebraic point of view, these coefficients are very important
because they have numerous interpretation in various field of mathematics:
\begin{itemize}
\item They count the multiplicity of induction or restriction of irreducible
  representations of the symmetric groups;
\item By Schur-Weyl duality, they also count the multiplicity of the tensor
  product of the irreducible representations of linear groups;
\item Geometry: mumber of intersection in a grassmanian variety, cup product
  of the cohomology;
\item Horn problem: eigenvalues of the sum of two hermitian matrix;
\item Extension of abelian groups (Hall algebra);
\item Application in quantum physics (spectrum rays of the Hydrogen atoms);
\end{itemize}


\section{Combinatorial Background}

\subsection{Ordered set}

  I'm using \SSR class/mixin/canonical paradigm.
\begin{coqcode}
Definition axiom T (r : rel T) :=
    [/\ reflexive r, antisymmetric r, transitive r &
        (forall m n : T, (r m n) || (r n m))].

Record mixin_of T := Mixin { r : rel T; x : T; _ : axiom r }.
Record class_of T := Class {base : Countable.class_of T; mixin : mixin_of T}.
Structure type := Pack {sort; _ : class_of sort; _ : Type}.
Notation ordType := type.

Definition leqX_op T := Order.r (Order.mixin (Order.class T)).

Delimit Scope ord_scope with Ord.
Open Scope ord_scope.
Notation "m <= n" := (leqX_op m n) : ord_scope.
\end{coqcode}

\subsection{Partitions}

different ways of decomposing an integer $n\in\N$ as a sum:
\[ 5=5=4+1=3+2=3+1+1=2+2+1=2+1+1+1=1+1+1+1+1 \]

Partition $\lambda := (\lambda_0\geq\lambda_1\geq\dots\geq\lambda_l > 0)$.\\
$|\lambda| := \lambda_0+\lambda_1+\dots+\lambda_l \qandq
\ell(\lambda) := l\,. $

Ferrer's diagram of a partitions : $(5,3,2,2) \leftrightarrow \yngs(0.5, 2,2,3,5)$

\begin{coqcode}
  Fixpoint is_part sh := (* Boolean Predicate *)
    if sh is sh0 :: sh'
    then (sh0 >= head 1 sh') && (is_part sh')
    else true.

  (* Boolean reflection lemma *)
  Lemma is_partP sh : reflect
    (last 1 sh != 0 /\ forall i, (nth 0 sh i) >= (nth 0 sh i.+1))
    (is_part sh).
\end{coqcode}

\subsection{Tableaux}

\begin{coqcode}
  Variable T : ordType.
  Notation Z := (inhabitant T).
  Notation is_row r := (sorted (@leqX_op T) r).

  Definition dominate (u v : seq T) :=
    ((size u) <= (size v)) && 
     (all (fun i => (nth Z u i > nth Z v i)%Ord) (iota 0 (size u))).

  Lemma dominateP u v :
    reflect ((size u) <= (size v) /\ 
             forall i, i < size u -> (nth Z u i > nth Z v i)%Ord)
            (dominate u v).

  Fixpoint is_tableau (t : seq (seq T)) :=
    if t is t0 :: t' then  
      [&& (t0 != [::]), is_row t0, 
        dominate (head [::] t') t0 & is_tableau t']
    else true.

  Definition to_word t := flatten (rev t).
\end{coqcode}

\subsection{Yamanouchi words}

  $\abs{w}_x$ : number of occurrence of $x$ in $w$.

  \begin{DEFN}
    Sequence $w_0,\dots,w_{(l-1}$ of integers such that for all $k, i$,
    \[ \abs{w_i,\dots,w_{l-1}}_k \geq \abs{w_i,\dots,w_{l-1}}_{k+1} \]

    Equivalently $(\abs{w}_i)_{i\leq\max(w)}$ is a partition and $w_1,\dots,w_{l-1}$ is
    also Yamanouchi.
  \end{DEFN}

  \[ (), 0, 00, 10, 000, 100, 010, 210, \]
  \[ 0000, 1010, 1100, 0010, 0100, 1000, 0210, 2010, 2100, 3210 \]

  \begin{coqcode}
(* incr_nth s i == the nat sequence s with item i incremented (s is *)
(*                 first padded with 0's to size i+1, if needed).   *)

  Fixpoint shape_rowseq s :=
    if s is s0 :: s'
    then incr_nth (shape_rowseq s') s0
    else [::].

  Definition shape_rowseq_count :=
    [fun s => [seq (count_mem i) s | i <- iota 0 (foldr maxn 0 (map S s))]].

  Lemma shape_rowseq_countE : shape_rowseq_count =1 shape_rowseq.

  Fixpoint is_yam s :=
    if s is s0 :: s'
    then is_part (shape_rowseq s) && is_yam s'
    else true.
\end{coqcode} 
% |

\section{The rule}

\begin{THEO}[Littlewood-Richardson rule]
  $c_{\lambda, \mu}^{\nu}$ is the number of (skew) tableaux of shape the
  difference $\nu/\lambda$, whose row reading is a Yamanouchi word of
  evaluation $\mu$.
\end{THEO}

% ...00 ...00 ...00
% ...1  ...1  ...1 
% .00   .01   .02  
% 12    02    01   

Some examples:
  \[
  C_{331,421}^{5432} = 3
  \qquad
  \Yboxdim{12pt}\scriptstyle
  \gyoung(12,:;00,:::;1,:::;00)\qquad
  \gyoung(02,:;01,:::;1,:::;00)\qquad
  \gyoung(01,:;02,:::;1,:::;00)
  \]
% ...00 ...00 ...00 ...00 ...00 ...00 ...00 ...00 ...00 ...00 ...00 ...00 ...00 ...00 ...00
% ...1  ...1  ...1  ...1  ...1  ...1  ...1  ...1  ...1  ...1  ...1  ...1  ...1  ...1  ...1
% .00   .00   .00   .01   .00   .01   .02   .02   .12   .02   .02   .01   .01   .12   .12
% 01    02    11    01    12    02    01    13    03    03    11    12    22    02    23

  \[
  C_{4321,431}^{7542} = 4
  \qquad
  \Yboxdim{12pt}\scriptstyle
  \gyoung(:;2,::;11,:::;01,::::;000)\quad
  \gyoung(:;1,::;12,:::;01,::::;000)\quad
  \gyoung(:;1,::;02,:::;11,::::;000)\quad
  \gyoung(:;0,::;12,:::;11,::::;000)\quad
  \]


  \def\AA{\red 0}
  \def\AB{\grn 1}
  \def\AC{\blu 2}
  \def\AD{{\color{pink} 3}}
  \[
  C_{431,4321}^{7542} = 4
  \qquad
  \Yboxdim{12pt}\scriptstyle
  \gyoung(\AC\AD,:;\AB\AB\AC,:::;\AA\AB,::::;\AA\AA\AA)\quad
  \gyoung(\AC\AD,:;\AA\AB\AC,:::;\AB\AB,::::;\AA\AA\AA)\quad
  \gyoung(\AB\AD,:;\AA\AC\AC,:::;\AB\AB,::::;\AA\AA\AA)\quad
  \gyoung(\AA\AD,:;\AB\AC\AC,:::;\AB\AB,::::;\AA\AA\AA)\quad
  \]

  \def\AA{\red 0}
  \def\AB{\grn 1}
  \def\AC{\blu 2}
  \def\AD{{\color{pink} 3}}
  \[
  C_{431,4322}^{7543} = 2
  \qquad
  \gyoung(\AB\AD\AD,:;\AA\AC\AC,:::;\AB\AB,::::;\AA\AA\AA)\quad
  \gyoung(\AA\AD\AD,:;\AB\AC\AC,:::;\AB\AB,::::;\AA\AA\AA)\quad
  \]

Coq statement:

\begin{coqcode}
Variable (d1 d2 : nat) (P1 : intpartn d1) (P2 : intpartn d2).
Variable (n : nat) (R : comRingType).
Hypothesis Hnpos : n != 0%N.
Notation Schur p := (Schur Hnpos R p).

Definition is_skew_reshape_tableau (x : seq nat) :=
  is_skew_tableau P1 (skew_reshape P1 P x).
Definition LRyam_set :=
  [set x : yamsh_finType (intpartnP P2) | is_skew_reshape_tableau x].
Definition LRyam_coeff := #|LRyam_set|.

Theorem LRtab_coeffP :
  Schur P1 * Schur P2 =
  \sum_(P : intpartn (d1 + d2) | included P1 P) Schur P *+ LRyam_coeff P.
\end{coqcode}

\end{document}

%%% Local Variables:
%%% compile-command: "pdflatex -shell-escape lrproof.tex"
%%% mode: latex
%%% TeX-master: t
%%% End:
