\documentclass[12pt,a4paper]{article}

\usepackage[utf8]{inputenc}
\usepackage[T1]{fontenc}
\usepackage[top=30pt,bottom=30pt,left=48pt,right=46pt]{geometry}

\usepackage{amsmath,amssymb,amsfonts,amsthm}
\usepackage{todonotes}
\usepackage{xspace}

\usepackage[vcentermath]{genyoungtabtikz}
% \YFrench % use french convention for tableaux.
%\usepackage{minted}
%\usemintedstyle{emacs}
%\usemintedstyle{colorful}
%\usemintedstyle{borland}
%\usemintedstyle{autumn}

%\newminted{coq}{
%frame=lines,
%framesep=2mm,
%mathescape=true
%}

\usepackage{listings}
\definecolor{dkblue}{rgb}{0,0.1,0.5}
\definecolor{lightblue}{rgb}{0,0.5,0.5}
\definecolor{dkgreen}{rgb}{0,0.4,0}
\definecolor{dk2green}{rgb}{0.4,0,0}
\definecolor{dkviolet}{rgb}{0.6,0,0.8}
\definecolor{brick}{rgb}{0.6,0.2,0.25}

% installation du mode SSR
\def\lstlanguagefiles{defManSSR.tex}
\lstset{language=SSR}
\let\verb=\lstinline

\usepackage{commath}

\newcommand{\Coq}{\texttt{Coq}\xspace}
\newcommand{\SSR}{\texttt{SSReflect}\xspace}
\newcommand{\MC}{\texttt{MathComp}\xspace}
\newcommand{\LR}{Littlewood-Richardson\ }
\newcommand{\var}[1]{\text{\verb{#1}}}

% INFO DOCUMENT - TITRE, AUTEUR, INSTITUTION
\title{\bf\LARGE A formal proof of \\
\LR rule\\[5mm]}
\author{Florent Hivert}
%\institute[LRI]{
%  LRI / Université Paris Sud 11 / CNRS / INRIA}
\date{Mai 2015}

\newcommand{\free}[1]{\left\langle#1\right\rangle}
\newcommand{\N}{{\mathbb N}}
\newcommand{\C}{{\mathbb C}}
\newcommand{\Z}{{\mathbb Z}}
\newcommand{\Q}{{\mathbb Q}}
\newcommand{\K}{{\mathbb K}}
\newcommand{\SG}{{\mathfrak S}}
\newcommand{\std}{\operatorname{Std}}
\newcommand{\sgn}{\operatorname{sgn}}
\newcommand{\comm}{\operatorname{comm}}
\newcommand{\sym}{\mathrm{sym}}
\newcommand{\NCSF}{\mathbf{NCSF}}
\newcommand{\QSym}{\mathrm{QSym}}
\newcommand{\FSym}{\mathbf{FSym}}
\newcommand{\freeS}{{\mathbb S}}

\newcommand{\partof}{\vdash}                    % Partition de
\newcommand{\compof}{\vDash}                    % Composisition de
\newcommand{\shape}{\operatorname{shape}} 

\newcommand{\qandq}{\text{\quad et\quad}}

\newcommand{\red}[1]{{\color{red} #1}}
\newcommand{\grn}[1]{{\color{green} #1}}
\newcommand{\blu}[1]{{\color{blue} #1}}

\newcommand{\alphX}{{\mathbb X}}
\newcommand{\alphY}{{\mathbb Y}}
\newcommand{\alphA}{{\mathbb A}}



\newtheorem{THEO}{Theorem}
\newtheorem{PROP}{Proposition}
\newtheorem{LEMMA}{Lemma}
\newtheorem{CORO}{Corollary}
\newtheorem{PROBLEM}{Problem}
\newtheorem{REMARK}{Remark}
\newtheorem{NOTE}{Note}

% \theoremstyle{definition}
\newtheorem{DEFN}{Definition}
\newtheorem{DEFNs}{Definitions}
\newtheorem{ALGO}{Algorithm}

\lstset{moredelim=[is][\color{red}\bfseries\ttfamily\underbar]{|*}{*|}}

%------------------------------------------------------------------------------
\begin{document}

\maketitle

\abstract{We present a formalized proof of the \LR rule using \Coq{} and
  \SSR{}. The \LR coefficients are defined as the coefficients of the
  expansion of the product of two Schur functions (the so-called structure
  constants). Recall that Schur functions form a particular basis of the ring
  of symmetric functions. The \LR coefficient are nonnegative integers and
  they have many algebraic interpretations, for example in geometry and group
  theory. The \LR rule allows to compute these coefficients as the number of
  filling of a specific shape with integer satisfying some comparison
  conditions. It is known that this way of computing those numbers is in some
  sense optimal.

  The proof follows more or less the Schützenberger argument as presented
  in~\cite{Lothaire}. It is based on an in depth study of a classical
  algorithm due to Schensted which compute the length of a longest increasing
  subsequence of a word. In particular, the central argument is a description
  of the output of the algorithm on the concatenation of two words knowing the
  output on those. Therefore, a typical feature of algebraic combinatorics,
  this is a proof of an algebraic identity, based on the understanding of the
  behavior of a algorithm.}

\tableofcontents

\section{Introduction}

\paragraph{Algebraic combinatorics}

\todo[inline]{rant about formalization of combinatorics}

Proof of algebraic identities by prooving properties of algorithms. Interplay
between algebra and algorithms.

\paragraph{Why formalizing sur a results}


We move on now to a general presentation of the problem. The goal is to give
the reader clearer idea of the field of algebraic combinatorics and the \LR
problem. In this first part we stay rather sketchy: to avoid to much
repetition, precise definitions will be given later together with their \Coq
formalization.

\subsection{Symmetric function and \LR coefficients}
The ring \emph{symmetric functions} is defined as a limit as $n$ goes to the
infinity of the ring of symmetric polynomial in $n$ indeterminates. This ring
serves as universal structure in which relations between symmetric polynomials
can be expressed in a way independent of the number $n$ of indeterminates (but
its elements are neither polynomials nor functions). Among other things, this
ring plays an important role in the representation theory of the symmetric
group or the general linear group~\cite{MacDo}. It also plays a role in
various geometric problems~\cite{Horn,Grassman}.

In these various context, the most important ingredient is a particular linear
basis $(s_\lambda)_\lambda$ whose elements are called the \emph{Schur
  functions}. Recall that linear basis of symmetric functions are indexed by
\emph{integer partition}, that is non increasing sequences of positive
integers. As a linear basis of an algebra, the product of two Schur functions
can be expressed as a linear combination of Schur functions:
\begin{equation}
  s_\lambda s_\mu = \sum_{\nu} C_{\lambda, \mu}^{\nu}\ s_\nu\,.
\end{equation}
For example,
\begin{multline}
  s_{(2,1)}\ s_{(4,2,2)} = s_{(4,2,2,2,1)} + s_{(4,3,2,1,1)} + s_{(4,3,2,2)} +
  s_{(4,3,3,1)} +
  s_{(4,4,2,1)} + s_{(4,4,3)} + \\
  s_{(5,2,2,1,1)} + s_{(5,2,2,2)} + 2s_{(5,3,2,1)} + s_{(5,3,3)} + s_{(5,4,2)}
  + s_{(6,2,2,1)} + s_{(6,3,2)}
\end{multline}
The coefficients $C_{\lambda, \mu}^{\nu}$ of the decomposition are called the
\LR coefficients and are nonnegative integers. The \LR rule describe them as
the number of certain combinatorial configurations called \LR tableaux. For
example, from the previous expansion one can read that
$C_{(2,1),(4,2,2)}^{(5,3,2,1)}=2$ which correspond to the two following
configurations:
\begin{equation}
  \gyoung(2,12,:;01,::;000)\qquad\qquad
  \gyoung(2,02,:;11,::;000)
\end{equation}
The precise definition of the configuration is rather intricate, crossing
several types of constraints on the filling of a diagram of box by numbers. It
makes the rule difficult to state, to use and even more to prove. Indeed,
according to Wikipedia~\cite{WikiLR}:
\begin{quotation}
  The \LR rule is notorious for the number of errors that appeared prior to
  its complete, published proof. Several published attempts to prove it are
  incomplete, and it is particularly difficult to avoid errors when doing hand
  calculations with it: even the original example in D.~E.~Littlewood and
  A.~R.~Richardson (1934) contains an error.
\end{quotation}

This rule was first stated in 1934 by D.~E.~Littlewood and
A.~R.~Richardson~\cite{LR}. However, their proof was wrong: they only proved
it in a very particular case. As already said, they also made a mistake in
their example. In 1938, Robinson~\cite{Robinson} attempted to complete the
proof, but he didn't quite fill the gap. One has to wait until 1977 to get the
first correct proof due to Schützenberger~\cite{SchutzLR}. This proof has
numerous combinatorial ingredient, and as it was written, some
combinatorialists though that this proof was ``somewhat gappy''. The present
work follows more or less this original proof, the algebraic step being closer
to the argument of \cite{NCSF7} showing that there where actually no crucial
gaps. After Schützenberger's first proof, one can find dozens of thesis and
paper about simplifying the argument~\cite{Zelevinsky81,Macdonald95,Gasharov98,DHT01,VanLeeuwen01,Stembridge02}
and the combinatorial study of these coefficients it is still a very active
research topic~(see for example~\cite{qAnalogs,KingToumazetTollu,KnutsonTao}
to only name a few).

\subsection{A computational point of view on the \LR rule}

From a computational point of view, such a rule give a good way to compute
those numbers. Indeed it was proved in 2006 by H.~Narayanan that the
computation of the \LR coefficients is
$\#P$-complete~\cite{Narayanan06}. Recall the $\#P$ is the complexity class
counting problem (i.e. with an answer in $\N$) analog of the complexity class
$NP$ for decision problem (i.e. with a boolean answer). More formally, $\#P$
is the class of function problems of the form "compute $f(x)$", where $f$ is
the number of accepting paths of a nondeterministic Turing machine running in
polynomial time. This roughly means that we shouldn't expect to have a better
algorithm to compute those number in general than enumerating the solution of
a combinatorial problem such as \LR tableaux.

Note that there are other combinatorial model for them such as Knutson and Tao
Honeycomb~\cite{KnutsonTao}. It is worth noting that the \LR coefficient
appear in Mulmuley's Geometric complexity theory, a strategy to prove that
$P\neq NP$ using invariant theory~\cite{Mulmuley} and algebraic geometry. In
particular he use Knutson and Tao's model to prove the surprising fact that
the problem of deciding the positivity of Littlewood-Richardson coefficients
belongs to $P$.

% \begin{quotation}
%   We point out that the remarkable Knutson and Tao Saturation Theorem and
%   polynomial time algorithms for LP have together an important and immediate
%   consequence in Geometric Complexity Theory. The problem of deciding
%   positivity of Littlewood-Richardson coefficients for GLn(C) belongs to
%   P. Furthermore, the algorithm is strongly polynomial.
% \end{quotation}

\subsection{Algebraic importance of \LR coefficients}

From the algebraic point of view, these coefficients are very important due to
their numerous interpretations. We present here briefly a few results, the
interested reader should refer to Fulton's overview~\cite{Fulton}.
\begin{itemize}
\item We defined them as the structure constants for the product in the ring
  of symmetric functions with respect to the basis of Schur functions;
  equivalently $C_{\lambda\mu}^\nu$ is the inner product $\langle s_\nu \mid
  s_\lambda s_\mu\rangle$; By some duality, this means that they also give the
  expansion of a Schur function on a union of two disjoint sets of variables:
  \begin{equation}
    s_\nu(\alphX\sqcup\alphY) = 
    \sum_{\lambda,\mu} C_{\lambda,\mu}^{\nu}\,
    s_\lambda(\alphX)\, s_\nu(\alphY)\,.
  \end{equation}
\item They count the multiplicity of induction or restriction of irreducible
  representations of the symmetric groups. More precisely $C_{\lambda\mu}^\nu$
  is the number of time the irreducible representation $V_\lambda \otimes
  V_\mu$ appear in the restriction of the representation $V_\nu$ of the
  symmetric group $\SG_{|\nu|}$ to the cartesian product $\SG_{|\lambda|}
  \times \SG_{|\mu|}$. By Frobenius reciprocity, this is also the number of
  times that $V_\nu$ occurs in the representation of $\SG_{|\nu|}$ induced
  from $V_\lambda \otimes V_\mu$.
\item By Schur-Weyl duality, they also count the multiplicity of the tensor
  product of the irreducible representations of linear groups or special
  linear groups:
  %\begin{equation}
    $E^\lambda \otimes E^\mu =\bigoplus_\nu (E^\nu)^{\oplus C_{\lambda\mu}^\nu}\,.$
  %\end{equation}
  \item They also have various geometrical interpretations: for example, they
    are the intersection number in a grassmanian variety and also appear in
    the cup product of its cohomology;
  \item They are connected to the horn problem: For any three partitions
    $\lambda,\mu,\nu$ of length at most $n$, the coefficient
    $C_{\lambda,\mu}^{\nu}$ is non zero if and only if there exists two real
    symmetric (resp. Hermitian, Hermitian quaternionic) matrices $A$ and $B$
    whose eigenvalues are $\lambda,\mu$ and such than the eigenvalues of $A+B$
    is $\nu$ (completing partitions with $0$ upto size $n$).
  \item They are related to extension of abelian $p$-groups (i.e. whose order
    is a power of a prime number $p$) through the Hall algebra: Upto
    isomorphism any abelian $p$-group is of the form
    $G_\lambda:=\Z/p^{\lambda_1}\Z\times\dots\times\Z/p^{\lambda_{l}}\Z$ for a
    unique partition $\lambda=\lambda_1,\dots,\lambda_{l}$ called its
    \emph{type}. The Hall coefficient $H_{\lambda,\mu}^{\nu}(p)$ is defined as
    the number of subgroup $F$ of $G_\nu$ of type $\lambda$ such that the
    quotient $G_\nu/F$ is of type $\mu$. Then $H_{\lambda,\mu}^{\nu}(p)$ is in
    fact a polynomial expression in $p$ whose leading coefficient is
    $C_{\lambda,\mu}^{\nu}$. In particular such a subgroup $F$ exists if and
    only if $C_{\lambda,\mu}^{\nu}$ is non zero.
\item Finally, their group theoretic nature gives them some application in
  quantum physics, when one compute the spectrum rays of the Hydrogen atoms.
\end{itemize}

\subsection{Outline of the paper}

\todo[inline]{present the plan}

\section{Formal proof background : \Coq and \SSR}

\todo[inline]{A small intro to \SSR}

Combinatorics is about case study and recursion.
Boolean reflexion allows to quickly solve the trivial cases

\section{Combinatorial Background : partitions, tableaux and Yamanouchi words}

In this first section we present the combinatorial ingredients together with
their formalization in \Coq. As far as we know, the present work is the first
formalization of those combinatorial objects.


\subsection{Ordered set}

The central ingredient of the proof is an algorithm due to
Schensted~\cite{schensted} which compute the length of a longest increasing
subsequence of a sequence on a totally ordered set called usually the
\emph{alphabet} and whose elements are called \emph{letters}. Though in the
proof, we only consider sequences of integers (or bounded integers), we decided
to formalize this algorithm in its full generality that is on any totally
ordered set. We therefore start by formalizing using \SSR mixin and canonical
paradigm the notion of totally ordered set.

There is however a small technical problem due to the fact that \Coq's doesn't
allows partial functions. Indeed, in many places we uses the \verb|nth|
function which compute the $n$-th element of a list. If $n$ is larger than the
size of the lists, the \verb|nth| function returns a default value which
must therefore be provided. Indeed \verb|nth| type is
\verb|forall T : Type, T -> seq T -> nat -> T|,
the default value is the first argument. We therefore decided to formalize
\emph{non empty totally ordered type} under the name \verb|ordtype|. Here
is the relevant part of the code (in the file \verb|ordtype.v|):
\begin{lstlisting}
Definition |*axiom*| T (r : rel T) :=
    [/\ reflexive r, antisymmetric r, transitive r &
        (forall m n : T, (r m n) || (r n m))].

Record |*mixin_of*| T := Mixin { r : rel T; x : T; _ : axiom r }.
Record |*class_of*| T := Class {base : Countable.class_of T; mixin : mixin_of T}.
Structure |*type*| := Pack {sort; _ : class_of sort; _ : Type}.
Notation |*ordType*| := type.
\end{lstlisting}
To ease readability we provide short notations in a specific scope with tag
\verb|Ord| for the comparison functions:
\begin{lstlisting}
Definition |*leqX_op*| T := Order.r (Order.mixin (Order.class T)).
Delimit Scope |*ord_scope*| with |*Ord*|.
Notation "m <= n" := (leqX_op m n) : ord_scope.
\end{lstlisting}
A witness of non-emptyness can be obtained using the \verb+inhabitant+ lemma:
\begin{lstlisting}
Lemma |*inhabitant*| (T : ordType) : T.
\end{lstlisting}
As an application, we provide \verb|nat| with a canonical \verb|ordtype|
structure which wraps the usual ordering. We provide \verb|0| as the witness
of non emptyness:
\begin{lstlisting}
Fact |*leq_order*| : Order.axiom leq.
[...]
Definition |*nat_ordMixin*| := Order.Mixin 0 leq_order.
Canonical |*nat_ordType*| := Eval hnf in OrdType nat nat_ordMixin.

Lemma |*leqXnatE*| m n : (m <= n)%Ord = (m <= n)%N.
\end{lstlisting}
The code further clone most of the \SSR \verb|nat |comparison lemmas. To
ease switching between \verb|nat| and arbitrary \verb|ordtype|, we
decided to add a letter \verb|X| to the comparison function names. For
example, the statement \verb|(m <= n) = ~~ (n < m)| is called
\verb|leqNgtn| for \verb|nat| and \verb|leqXNgtnX| for
arbitrary \verb|ordtype|. We give further constructions such as dual
order. The file also prove a bunch of useful lemmas about sequences of elements
belonging to an ordered type dealing for example with maximum element an
particularly its last occurrence.

\subsection{Partitions}

The theory of symmetric function is closely related to partition theory. They
are defined as the different ways of decomposing an integer $n\in\N$ as a sum:
\[ 5=5=4+1=3+2=3+1+1=2+2+1=2+1+1+1=1+1+1+1+1\,. \] Two decompositions that
differ only by their order are considered equal. To ensure unicity of the
machine representation, we follow the usual convention to sort the summand
(called \emph{part}) in decreasing order. In \SSR this also ensure that
equality of partition is the same as Leibnitz equality of the corresponding
lists.
\begin{DEFN}
  A \emph{partition} $\lambda$ of an integer $n$ is a finite decreasing
  sequence of positive integers
  $(\lambda_0\geq\lambda_1\geq\dots\geq\lambda_{l-1} > 0)$ whose sum is
  $n$. We denote $|\lambda| := n = \lambda_0+\lambda_1+\dots+\lambda_{l-1}$
  the sum and $\ell(\lambda) := l$ the length. We also denote $\lambda\partof
  n$ the fact that $\lambda$ is a partition of $n$.

  Conventionally, that there is only one partition of the integer $0$ namely
  the empty sequence so that $\lambda\partof0$ means that $\lambda = ()$.
\end{DEFN}
We decided to represent partitions naturally by terms of type 
\verb+seq nat+.
Following \SSR paradigm, the definition \verb|is_part| is given as
a computational boolean predicate (a recursive function). Since a
\verb{fixpoint} is not always the easiest statement to use is a proof, we
provide two equivalent statements (in \verb|Prop|) using two reflection
lemmas. These equivalent statement are also perhaps easier to read:
\begin{lstlisting}
  Fixpoint |*is_part*| sh := (* Boolean Predicate *)
    if sh is sh0 :: sh'
    then (sh0 >= head 1 sh') && (is_part sh')
    else true.

  (* Boolean reflection lemmas *)
  Lemma |*is_partP*| sh : reflect
    (last 1 sh != 0 /\ forall i, (nth 0 sh i) >= (nth 0 sh i.+1))
    (is_part sh).

  Lemma |*is_part_ijP*| sh : reflect
    (last 1 sh != 0 /\ forall i j, i <= j -> (nth 0 sh i) >= nth 0 sh j)
    (is_part sh).
\end{lstlisting}
The set $P_n$ of partition of a given $n$ is finite, so we provide a function
\verb|enum_partn| for enumerating them. The following lemmas assert that
\verb|enum_partn sm| if a complete and duplicate free list of all the
partitions of $n$:
\begin{lstlisting}
Definition |*is_part_of_n*| sm := [pred p | (sumn p == sm) & is_part p ].
Definition |*enum_partn*| sm := [...]
Lemma |*enum_partn_allP*| sm : all (is_part_of_n sm) (enum_partn sm).
Lemma |*enum_partn_countE*| sm p : is_part_of_n sm p -> count_mem p (enum_partn sm) = 1.
\end{lstlisting}
Thanks to this list, we further model the finite set $P_n$ by defining a
dependant type and providing it with a canonical \SSR's \verb+fintype+
structure.
\begin{lstlisting}
Structure |*intpartn*| n : predArgType :=
  IntPartN {pnval :> seq nat ; _ : is_part_of_n n pnval}.
[...]
Let |*type*| := sub_finType intpartn_subCountType (enum_partn_allP n) (@enum_partn_countE n).
Canonical |*intpartn_finType*| := Eval hnf in [finType of intpartn for type].

\end{lstlisting}
Note that by design choice, most of the lemmas on partition (and other
combinatorial object such as tableaux, Yamanouchi words) require a
\verb|seq nat| together with a proof of \verb+is_part+ rather that some
dependant type. Dependant type are only used when statements need to confine a
set in a \verb+fintype+ e.g. statement about cardinalities.

It is customary to depict a partition by a diagram of boxes called its Ferrers
diagram. Namely the Ferres diagram of a partition $\lambda := (\lambda_0,
\lambda_1,\dots,\lambda_{l-1})$ is obtained by piling left justified rows of
boxes of respective length $\lambda_0, \lambda_1,\dots,\lambda_{l-1}$. We use
the French convention which put the longest row at the bottom of the
picture (English literature usually draw them upside down). For example,
\[(7,5,3,2,2)\quad\text{ is depicted as }\quad \yngs(0.5, 2,2,3,5,7).\]
\bigskip

Partition are ordered by the inclusion of their diagram. Here is the
corresponding boolean predicate and the reflection lemma:
\begin{lstlisting}
Fixpoint |*included*| inner outer :=
  if inner is inn0 :: inn then
    if outer is out0 :: out then
      (inn0 <= out0) && (included inn out)
    else false
  else true.

Lemma |*includedP*| inner outer : reflect
  (size inner <= size outer /\ forall i, nth 0 inner i <= nth 0 outer i)
  (included inner outer).
\end{lstlisting}
A skew partition is the difference of two included partitions:
\[(7,5,3,2,2) / (4,2,1)\quad\leftrightarrow\quad \gyoungs(0.5,\ \ ,\ \ ,:;\ \
,::;\ \ \ ,::::;\ \ \ ).\] We didn't define a specific (dependant) type for
skew partition in \Coq: when a skew partition is required we just pass proof
of \verb+is_part inner+, \verb+is_part outer+ together with
\verb+included inner outer+.

\subsection{Tableaux}

Tableaux were invented by Alfred Young to understand the representation of the
symmetric groups. They are the central character of the story told here.
\begin{DEFN}
  Let $\alphA$ be an alphabet (that is a totally ordered set). A \emph{Young
    tableau} or \emph{tableau} for short is a filling $T$ with letters from
  $\alphA$ of the diagram of a partition $\lambda$ which is emph{non
    decreasing along rows} and \emph{strictly increasing along columns}. The
  partition $\lambda$ is called the \emph{shape} of $T$.

  A \emph{standard tableau} is a tableau over the integer such that each
  integer between $0$ and $n-1$ where $n$ is the sum of the shape appear only
  once.

  A \emph{skew tableau} is a tableau whose shape is a skew shape.
\end{DEFN}
Note that in the literature, standard tableau are usually labeled from $1$ to
$n$. Here is an example of a tableau, a standard tableau and a skew tableau.
\[
  \young(ff,cdd,bccdf,aabeefgh)\qquad
  \young(7,4,258,01369)\qquad
  \gyoung(12,:;00,:::;1,:::;00)\qquad
\]
Following~\cite{Lothaire}, we formalize tableau by defining a \emph{dominance}
relation between two consecutive nondecreasing sequence called \emph{rows}:
\begin{DEFN}
Let $\alphA$ be an alphabet (that is a totally ordered set). 
  A non decreasing word $v \in \alphA^*$ is called a \emph{row}. Let $u = x_0
  \dots x_{r-1}$ and $v = y_0 \dots y_{s-1}$ be two rows. ($x_i, y_j \in \alphA$). We
  say that \emph{$u$ dominates $v$} if $r\leq s$ and for $i = 0,\dots,r-1$,
  $x_i > y_i$.

  A \emph{tableau} is a sequence of non empty row which is reverse sorted for
  the dominance order.
\end{DEFN}
As for partitions, we don't use a specific type for rows and tableaux and use
simply \verb{seq seq nat} as a data-structure. Tableau are just defined
using a predicate \verb{is_tableau}. We show here only the boolean
reflection lemmas, and refer the reader to the file for the formal definition
of standard tableau and skew tableau.
\begin{lstlisting}
Variable T : ordType.
Notation Z := (inhabitant T).
Notation |*is_row*| r := (sorted (@leqX_op T) r).

Lemma |*dominateP*| u v : reflect
  ((size u) <= (size v) /\ forall i, i < size u -> (nth Z u i > nth Z v i)%Ord)
  (dominate u v).

Lemma |*is_tableauP*| t : reflect
  [/\ forall i, i < size t -> (nth [::] t i) != [::],  (* forbid empty rows *)
     forall i, is_row (nth [::] t i) &
     forall i j, i < j -> dominate (nth [::] t j) (nth [::] t i) ]
  (is_tableau t).
\end{lstlisting}

A very important notion is the row reading of a tableau. It is just the word
obtained from the natural reading (top to bottom and left to right) of a
tableau. For example, the reading of the first tableau above is:
$ffcddbccdfaabeefgh$.
It is defined in coq by
\begin{lstlisting}
Definition |*to_word*| t := flatten (rev t).
\end{lstlisting}

Many of the reasoning involved in the present work involve surgery on
tableaux. For example, we define a function \verb{join_tab} which glues
the line of two (possibly skew) tableaux, as exemplified below
\begin{lstlisting}
Definition |*join_tab*| s t := [seq r.1 ++ r.2 | r <- zip (pad [::] (size t) s) t].
\end{lstlisting}
\[
\var{join_tab}\left(
  \ \young(,c,bcc,aab)\ ,\ 
  \gyoung(ff,:;dd,:::;df,:::;eefgh)\ \right)\ =\
  \young(ff,cdd,bccdf,aabeefgh)\qquad
\]
As an example of surgery, the following lemma assert that if all the entries
of a tableau \verb{s} are smaller than all the entries of a skew tableau
\verb{t} whose inner shape is the shape of \verb{s} then the join of
\verb{s} and \verb{t} is a tableau:
\begin{lstlisting}
Lemma |*join_tab_skew*| s t :
  all (allLtn (to_word s)) (to_word t) ->
  is_tableau s -> is_skew_tableau (shape s) t ->
  is_tableau (join_tab s t).
\end{lstlisting}
Interestingly, though any combinatorialist would consider that such a lemma is
so trivial that is doesn't require any proof, the formal proof is 58~lines
long.

\subsection{Yamanouchi words}

Yamanouchi is a data structure which is equivalent (meaning that there is a
simple bijection) to standard tableaux. However, since they are one
dimensional, they are sometimes easier to manipulate. They also appears in the
\LR rule statement.

For a word $w$ we write $\abs{w}_x$ the number of occurrence of $x$ in
$w$. This is computed in \SSR by \verb{count_mem x w}.
\begin{DEFN}
  A word $w := w_0,\dots,w_{l-1}$ over the integers is \emph{Yamanouchi} if for
  all $k, i \in \N$,
  \[ \abs{w_i,\dots,w_{l-1}}_k \geq \abs{w_i,\dots,w_{l-1}}_{k+1} \]
\end{DEFN}
As we code Yamanouchi words by term of type \verb{seq nat}, it is easier
to have a recursive definition. Here it is:
\begin{DEFN}
  Equivalently, $w$ is \emph{Yamanouchi} either if its empty or if
  $(\abs{w}_i)_{i\leq\max(w)}$ (called the row-shape of $w$) is a partition
  and $w_1,\dots,w_{l-1}$ is also Yamanouchi.
\end{DEFN}
Here is a complete list of Yamanouchi word of length smaller than $4$.
\begin{gather*}
  (), 0, 00, 10, 000, 100, 010, 210, \\
  0000, 1010, 1100, 0010, 0100, 1000, 0210, 2010, 2100, 3210
\end{gather*}
The partition $(\abs{w}_i)_{i\leq\max(w)}$ is denoted \verb|evalseq w|.
 We recall that in \SSR, the call \verb{incr_nth s i} design the nat
sequence \verb|s| with item $i$ incremented (first padded with 0's to
size $i+1$, if needed).

  \begin{lstlisting}
Fixpoint |*evalseq*| s :=
  if s is s0 :: s'
  then incr_nth (evalseq s') s0
  else [::].
Definition |*evalseq_count*| := (* iterative version *)
  [fun s => [seq (count_mem i) s | i <- iota 0 (foldr maxn 0 (map S s))]].
Lemma |*evalseq_countE*| : evalseq_count =1 evalseq.

Fixpoint |*is_yam*| s :=
  if s is s0 :: s'
  then is_part (evalseq s) && is_yam s'
  else true.

Lemma |*is_yamP*| s : reflect
  (forall i n, count_mem n (drop i s) >= count_mem n.+1 (drop i s))
  (is_yam s).
\end{lstlisting}

For standard tableau as well as Yamanouchi word, we finally provide, as we
already explained for partition, enumeration functions. This allows use is to
endow the various dependant type (e.g. for size or shape), with a canonical
\verb{fintype} structure (see for example \verb{is_yam_of_shape},
\verb{enum_yameval}, \verb{yameval}, \verb{yameval_finType} for
Yamanouchi words of a given row-shape.
\todo{Do I want to present the bijection with standard tableau here ?}
\section{Schur polynomial and the \LR rule}

The \LR rule allows to expand product of symmetric polynomials in a particular
basis called Schur polynomials. The theory of symmetric polynomial is usually
carried trough their limit as the number of variable tends to the infinity
which are called symmetric function (though they are neither polynomial nor
function but rather bounded degree formal power series). We provide here an
extremely short introduction. The interested reader should refer to Macdonald
books~\cite{Macddo}.

\subsection{Symmetric functions and Schur function}

We fix a positive integer $n$ and consider polynomials in the set of variables
$\alphX_n:=\{x_0,\dots,x_{n-1}\}$. The symmetric group $\SG_n$ on
$\{0,\dots,n-1\}$ acts on variable by permutation.
\begin{DEFN}
 A polynomial $f(x_0,\dots,x_{n-1})$ is symmetric if it is invariant by any
 permutation of the variables, that is for all permutation $\sigma\in\SG_n$,
 \begin{equation}
   f^\sigma(\alphX) := f(x_{\sigma(0)}, x_{\sigma(1)}, \dots, x_{\sigma(n-1)}) = 
   f(x_{0}, x_{1}, \dots, x_{n-1})\,.
 \end{equation}
\end{DEFN}
The sum of two symmetric polynomial is symmetric as well as the product of a
symmetric polynomial by a scalar or the product of two symmetric
polynomial. Therefore they form a sub-algebra denoted $\sym(\alphX_n)$ of the
algebra of polynomials.

A natural basis of this algebra is given by the so-called monomial symmetric
polynomials. They are defined as the sum of the orbit of a monomial under the
action of the symmetric group. Since in any orbit there is only one monomial
whose exponents are sorted decreasingly, it makes sense to index the element
by this sequence of exponents. Removing the zeroes at the end of this sequence
gives a partition of length at most $n$. Therefore the basis of symmetric
polynomials are indexed by partition of length at most $n$. Here are two
examples of monomial symmetric polynomials:
\begin{equation*}
  m_{(2,1)}(x_0,x_1,x_2) =
  x_0^2x_1 + x_0x_1^2 + x_0^2x_2 + x_1^2x_2 + x_0x_2^2 + x_1x_2^2\,.
\end{equation*}
\begin{multline*}
  m_{(2,2,1)}(x_0,x_1,x_2,x_3) =
  x_0^2x_1^2x_2 + x_0^2x_1x_2^2 + x_0x_1^2x_2^2 + x_0^2x_1^2x_3 +
  x_0^2x_2^2x_3 + x_1^2x_2^2x_3 + \\
  x_0^2x_1x_3^2 + x_0x_1^2x_3^2 + x_0^2x_2x_3^2 + x_1^2x_2x_3^2 +
  x_0x_2^2x_3^2 + x_1x_2^2x_3^2\,.
\end{multline*}
However, it appear in the theory that this simple basis has not that much
interesting properties. On the contrary, one of the most important basis is
the so-called \emph{Schur polynomials} They were first defined by Jacobi but
they are named in the honor of Schur who discovered their importance in the
representation theory. The original definition of Jacobi is the following:
\begin{DEFN}[Jacobi's definition of Schur function]
  let $\lambda:=(\lambda_0\dots\lambda_{\ell-1})$ be a partition of length
  $\ell$ at most $n$. We complete $\lambda$ by setting $\lambda_i:=0$ whenever
  $i\geq\ell$. Then
  \begin{equation}
    s_{\lambda} (\alphX) :=
    \frac{1}{\Delta}\sum_{\sigma\in\SG_n}
    \sgn(\sigma)\ 
    (x_0^{\lambda_0 + n-1}x_1^{\lambda_1 + n-2}\cdots x_{n-1}^{\lambda_{n-1}})^\sigma
  \end{equation}
  where
  \begin{itemize}
  \item $\Delta:=\prod_{0\leq i<j<n} (x_i - x_j)$ is the Vandermonde
    determinant;
  \item $\sgn(\sigma)$ is $+1$ or $-1$ whether $\sigma$ is an
    even or odd permutations (the parity of the number of transpositions in a
    decomposition of $\sigma$).
  \end{itemize}
\end{DEFN}
Note that the numerator sum is antisymmetric, meaning that it is multiplied by
$\sgn(\mu)$ under the action of a permutation $\mu$.  This implies that the
sum is divisible by $\Delta$ so that $s_\lambda$ is a proper polynomial. As
the quotient of two antisymmetric polynomials it is moreover
symmetric. Furthermore it is simple to see that
\begin{equation}
s_{\lambda} (\alphX) =
   x_0^{\lambda_0}x_1^{\lambda_1}\cdots x_{n-1}^{\lambda_{n-1}} +  \cdots
\end{equation}
where the rest of the terms contains only monomial which are larger for the
reverse lexicographic order on the exponents. As a consequence, the Schur
polynomials are linearly independant and form a \emph{basis of the algebra of
  symmetric polynomials}.

Interestingly enough, \emph{we did not need to formalize} this part of the
theory. Indeed, we decide to chose a definition which has a more combinatorial
feeling. This may looks like a kind of cheating. However, the equivalence of
this definition with the combinatorial one is generally proved showing that
both verify a recursive rule (due to Pieri) which give the product of a Schur
polynomial by an elementary one. This rule is a very particular case of the
\LR rule, so that it didn't really simplifies the proof. Also most of the
paper dealing with the \LR rule start from the combinatorial definition.
\begin{DEFN}[Combinatorial definition of Schur function]

  let $\lambda:=(\lambda_0\dots\lambda_{\ell-1})$ be a partition of length
  $\ell$ at most $n$. For a tableau $t$ over the alphabet $\{0,\dots,n-1\}$ we
  denote $\alphX^t$ the product $\prod_{i\in t}x_i$ (equivalently it is the
  image of the row reading of $t$ under the morphism $i\mapsto x_i$ which
  sends non-commutative words to commutative monomials). Then
  \begin{equation}
    s_\lambda(\alphX) := \sum_{t\ \mid\ \shape(t) = \lambda}  \alphX^\lambda\,.
  \end{equation}
  the sum being taken over all the tableau of shape $\lambda$.
\end{DEFN}
For example,
\begin{gather*}
  s_{(2,1)}(x_0,x_1,x_2) =
  \begin{array}[b]{c@{\ +\ }c@{\ +\ }c@{\ +\ }c@{\ +\ }c@{\ +\ }c@{\ +\ }l}
    % \Yboxdim{8pt}\scriptsize
    \young(1,00) & \young(1,01) & \young(2,00) &
    \young(1,02)\ \young(2,01) & \young(2,11) & \young(2,02) & \young(2,12) \\[5mm]
    x_0^2x_1 & x_0x_1^2 & x_0^2x_2 & 2\,x_0x_1x_2 & x_1^2x_2 & x_0x_2^2 & x_1x_2^2\,.
  \end{array}
\end{gather*}
Though it is non trivial in general, one can check on this example that the
result is indeed a symmetric polynomial. Indeed, one can subsume the previous
example by
\begin{equation*}
  s_{(2,1)} = 2\,m_{(1,1,1)} + m_{(2,1)}\,.
\end{equation*}
Another example is
\begin{equation}\label{eq:example_schur}
  s_{3211} = 
  35\,m_{1111111} + 15\,m_{211111} + 6\,m_{22111} + 2\,m_{2221} + 3\,m_{31111} + m_{3211}\,.
\end{equation}
The coefficient $6$ for $m_{22111}$ that is the coefficient of
$x_0^2x_1^2x_2x_3x_4$ is the number of tableau of shape $(3,2,1,1)$ whose row
reading is a permutations of $0011234$. Here is the list
\begin{equation*}
  \young(4,3,12,001)\qquad\young(4,2,13,001)\qquad\young(3,2,14,001)\qquad
  \young(4,3,11,002)\qquad\young(4,2,11,003)\qquad\young(3,2,11,004)\,.
\end{equation*}
The fact that $s_{3211}$ is symmetric express for example that the coefficient
of $x_0^2x_1^2x_2x_3x_4$ is also the coefficient of $x_0x_1^2x_2x_3x_4^2$ as
one can check on the following list of tableaux
\begin{equation*}
  \young(4,3,24,011)\qquad\young(4,3,14,012)\qquad\young(4,3,12,014)\qquad
  \young(4,2,13,014)\qquad\young(4,2,14,013)\qquad\young(3,2,14,014)
\end{equation*}
Again this is non trivial that these two set of tableaux are equinumerous.
\smallskip

Another very important point is that all those computation are independant of
the number of variables. More precisely, in the preceding example
(Equation~\ref{eq:example_schur}), this equation is true for any number of
variables, the difference is that $s_\lambda(\alphX_n)$ and
$m_\lambda(\alphX_n)$ vanishes if $\ell(\lambda) > n$.
\bigskip

\MC defines univariate polynomials over an arbitrary commutative ring
\verb{R}. So to define multivariate polynomials, we use recursive polynomials:
$\var{R}[x_0,\dots,x_{n+1}]:=\var{R}[x_0,\dots,x_n][x_{n+1}]$. Recall
that \verb{'I_n} denote in \SSR the finite subType of integers
$i < \var{n}$.
\begin{lstlisting}
Fixpoint |*multpoly*| n :=
  if n is n'.+1 then poly_comRingType (multpoly n') else R.
Fixpoint |*vari*| n : 'I_n -> multpoly n := [...] (* The variable $x_n$ *)
\end{lstlisting}
Then to formalize Schur polynomials, we need the commutative image of a word,
and the sum of the commutative image of a set a words of a given length
\verb{d}:
\begin{lstlisting}
Definition |*commword*| (w : seq 'I_n) : multpoly n := \prod_(i <- w) vari i.
Definition |*polyset*| d (s : {set d.-tuple 'I_n}) := \sum_(w in s) commword w.
\end{lstlisting}
An important remark is that once the shape is fixed, on can recover any
(skew)-tableau from its row reading. We therefore define Schur function
$s_\lambda$ as the commutative image of the set of the row-reading of tableaux
of shape $\lambda$. Again there is an decision procedure for being such a
row-reading which is not shown here:
\begin{lstlisting}
Definition |*is_tableau_of_shape_reading*| (sh : seq nat) (w : seq A) := [...]
Lemma |*is_tableau_of_shape_readingP*| (sh : seq nat) (w : seq A) : reflect
  (exists tab, [/\ is_tableau tab, shape tab = sh & to_word tab = w])
  (is_tableau_of_shape_reading sh w).

Definition |*tabwordshape*| (sh : intpartn d) :=
  [set t : d.-tuple 'I_n | is_tableau_of_shape_reading sh t ].

Definition |*Schur*| d (sh : intpartn d) := polyset R (tabwordshape sh).
\end{lstlisting}
\todo{Comment about tuple vs lists and finset}
\subsection{The statement of the rule}

At last we come to the statement of the rule. They express the coefficient of
the product (the so called structure constants) of the symmetric polynomials
in the basis of Schur polynomials.
\begin{DEFN}
  Let $A$ be a algebra over a field $\K$ with a basis $B := (b_i)_{i\in I}$
  indexed by a set $I$. The \emph{structure constants of $A$ in the basis $B$}
  are the coefficients $C_{i,j}^k$ with $i,j,k\in I$ of the expansion of the
  product $b_i b_j= \sum_{k\in I} C_{i,j}^k b_k$.

  The \emph{\LR coefficients} $C_{\lambda, \mu}^{\nu}$ are the structure
  constants of the algebra of symmetric polynomials in the basis of Schur
  polynomials.
\end{DEFN}
The \LR rule states that the \LR coefficients are non-negative integer, and
give a way to compute them:
\begin{THEO}[Littlewood-Richardson rule]\label{theo:LR-rule}
  $C_{\lambda, \mu}^{\nu}$ is the number of (skew) tableaux of shape the
  difference $\nu/\lambda$, whose row reading is a Yamanouchi word of
  evaluation $\mu$.
\end{THEO}
% ...00 ...00 ...00
% ...1  ...1  ...1 
% .00   .01   .02  
% 12    02    01

Here are some examples:

  \def\AA{\red 0}
  \def\AB{\grn 1}
  \def\AC{\blu 2}
  \def\AD{{\color{gray} 3}}
  \[
  C_{331,\red4\grn2\blu1}^{5432} = 3
  \qquad
  \Yboxdim{12pt}\scriptstyle
  \gyoung(\AB\AC,:;\AA\AA,:::;\AB,:::;\AA\AA)\qquad
  \gyoung(\AA\AC,:;\AA\AB,:::;\AB,:::;\AA\AA)\qquad
  \gyoung(\AA\AB,:;\AA\AC,:::;\AB,:::;\AA\AA)
  \]

  \[
  C_{4321,\red4\grn3\blu1}^{7542} = 4
  \qquad
  \Yboxdim{12pt}\scriptstyle
  \gyoung(:;\AC,::;\AB\AB,:::;\AA\AB,::::;\AA\AA\AA)\quad
  \gyoung(:;\AB,::;\AB\AC,:::;\AA\AB,::::;\AA\AA\AA)\quad
  \gyoung(:;\AB,::;\AA\AC,:::;\AB\AB,::::;\AA\AA\AA)\quad
  \gyoung(:;\AA,::;\AB\AC,:::;\AB\AB,::::;\AA\AA\AA)
  \]

  \[
  C_{431,\red4\grn3\blu2\color{gray}1}^{7542} = 4
  \qquad
  \Yboxdim{12pt}\scriptstyle
  \gyoung(\AC\AD,:;\AB\AB\AC,:::;\AA\AB,::::;\AA\AA\AA)\ 
  \gyoung(\AC\AD,:;\AA\AB\AC,:::;\AB\AB,::::;\AA\AA\AA)\ 
  \gyoung(\AB\AD,:;\AA\AC\AC,:::;\AB\AB,::::;\AA\AA\AA)\ 
  \gyoung(\AA\AD,:;\AB\AC\AC,:::;\AB\AB,::::;\AA\AA\AA)
  \]

  \[
  C_{431,\red4\grn3\blu2\color{gray}2}^{7543} = 2
  \qquad
  \gyoung(\AB\AD\AD,:;\AA\AC\AC,:::;\AB\AB,::::;\AA\AA\AA)\quad
  \gyoung(\AA\AD\AD,:;\AB\AC\AC,:::;\AB\AB,::::;\AA\AA\AA)
  \]

Since once the shape is fixed, one can recover the tableau from its row
reading, the statement can be rephrased as
\begin{THEO}[Littlewood-Richardson rule]
  $C_{\lambda, \mu}^{\nu}$ is the number of Yamanouchi word of evaluation
  $\mu$ which are the row reading of a (skew) tableaux of shape the difference
  $\nu/\lambda$.
\end{THEO}

In \SSR, to be able to speak of the cardinality of a set, we must live inside
a finite type (\verb{fintype}). We provide the dependant type of Yamanouchi
words of evaluation $\mu$ with a canonical \verb{fintype} namely
\verb{yameval_finType}. So to be able to write a such a statement we switch
using dependant type.
\begin{lstlisting}
Variable (d1 d2 : nat) (P1 : intpartn d1) (P2 : intpartn d2).
Hypothesis |*Hnpos*| : n != 0%N. (* ensure that the alphabet is not empty *)
Notation |*Schur*| p := (Schur Hnpos R p).
\end{lstlisting}
The following definitions and reflection lemma formalize the property of being
the row-reading of a skew tableau of shape $\var{P}/\var{P1}$:
\begin{lstlisting}
Definition |*is_skew_reshape_tableau*| P P1 (w : seq nat) :=
  is_skew_tableau P1 (skew_reshape P1 P w).
Lemma |*is_skew_reshape_tableauP*| P P1 (w : seq nat) :
  size w = sumn (diff_shape P1 P) -> reflect
    (exists tab, [/\ is_skew_tableau P1 tab,
                     shape tab = diff_shape P1 P & to_word tab = w])
    (is_skew_reshape_tableau P P1 w).
\end{lstlisting}
Now the \LR coefficient $X_{\var{P1},\var{P2}}^{\var{P}}$ is the cardinality of
the set of Yamanouchi word of evaluation which are row reading of tableau of
shape $\var{P}/\var{P1}$:
\begin{lstlisting}
Definition |*LRyam_set*| :=
  [set x : yameval_finType (intpartnP P2) | is_skew_reshape_tableau P P1 x].
Definition |*LRyam_coeff*| := #|LRyam_set|.
\end{lstlisting}
So that the \LR rule statement is:
\begin{lstlisting}
Theorem |*LRtab_coeffP*| :
  Schur P1 * Schur P2 =
  \sum_(P : intpartn (d1 + d2) | included P1 P) Schur P *+ LRyam_coeff P.
\end{lstlisting}
Due to the use of dependant type, Coq is not able to compute a coefficient
from this definition. We provide a dependant type free definition which allows
computation:
\begin{lstlisting}
Definition LRyam_enum (P1 P2 P : seq nat) :=
  [seq x <- enum_yameval P2 | is_skew_reshape_tableau P P1 x].
Definition LRyam_compute (P1 P2 P : seq nat) := size (LRyam_enum P1 P2 P).
Lemma LR_coeff_computeP : LRyam_compute P1 P2 P = LRyam_coeff.
\end{lstlisting}
Then
\begin{lstlisting}
Eval compute in (LRyam_enum  [:: 4; 3; 1] [:: 4; 3; 2; 1] [:: 7; 5; 4; 2]).
     = [:: [:: 0; 3; 1; 2; 2; 1; 1; 0; 0; 0];
           [:: 1; 3; 0; 2; 2; 1; 1; 0; 0; 0];
           [:: 2; 3; 0; 1; 2; 1; 1; 0; 0; 0];
           [:: 2; 3; 1; 1; 2; 0; 1; 0; 0; 0]]
     : seq (seq nat)
Eval compute in (LRyam_compute  [:: 4; 3; 1] [:: 4; 3; 2; 1] [:: 7; 5; 4; 2]).
     = 4
     : nat
\end{lstlisting}
Note that this is not an efficient ways of computing them. On cannot get
coefficients larger than 10 in a reasonable time using this definition. A more
efficient way of computing them will be presented in Section~\ref{implem}.

\subsection{The idea of the proof: the noncommutative lifting}

The main idea of the proof is to go to non-commutative polynomials that is to
the free algebra. The \LR rule we be obtained as a commutative image of the
so-called free \LR rule. We already defined Schur polynomials as some kind of
commutative image. Indeed, the definition rewrites as
\begin{equation}
  s_\lambda = \sum_{w \in \operatorname{RRT}(\lambda)}
  \comm(w) = \comm\left(\sum_{w \in \operatorname{RRT}(\lambda)} w\right)
\end{equation}
where $\operatorname{RRT}(\lambda)$ is the set of row-reading of the tableau
of shape $\lambda$. The main point is to remark that
$\operatorname{RRT}(\lambda)$ is not the only set whose commutative image is
$s_\lambda$. Using a algorithm due to Schensted one can construct a map $Q$
from word to standard tableau such that for all standard tableau $t$
\begin{equation}
  \comm\left(\sum_{w\, \mid\, Q(w)=t} w\right) = s_{\shape(t)}\,.
\end{equation}
We call the sets $\freeS_t := \{w \mid Q(w)=t\}$ the \emph{free Schur
  functions} (One usually rather give this name to the formal sum of its
element as a lifting of the Schur function in the free algebra, whence the
name).  Now the key fact is that the concatenation of two free Schur functions
(that is the product in the free algebra) is a union of free Schur functions:
\begin{THEO}[Free \LR rule]\label{theo:free-LR_rule}
  For any two standard tableau $c$, $d$ there exists a set $\operatorname{LR}(c,
  d)$ such that
  \begin{equation}
    \label{eq:free-lr-rule}
    \freeS_c\cdot\freeS_d := \{ u\cdot v \mid u\in \freeS_c, u\in \freeS_d \} =
    \bigsqcup_{t\in\operatorname{LR}(c, d)} \freeS_t\,.
  \end{equation}
\end{THEO}
Now since the commutative image of a concatenation is the product of
commutative image, taking the commutative image of the previous equality gives that
\begin{equation}
  \comm(\freeS_c\cdot\freeS_d) = s_{\shape(c)}s_{\shape(d)} =
  \sum_{t\in\operatorname{LR}(c, d)} s_{\shape(t)}\,.
\end{equation}
Otherwise said the \LR coefficients can be expressed as
\begin{THEO}[Tableau version of the Littlewood-Richardson rule]
  \begin{equation}
    C_{\lambda,\mu}^{\nu} = \#\{t\in\operatorname{LR}(c, d) \mid \shape(t) = \nu\}\,,
  \end{equation}
  where $c$ (resp. $d$) is any standard tableau of shape $\lambda$
  (resp. $\mu$).
\end{THEO}
This indeed greatly simplify the computation because the equality of
Equation~\ref{eq:free-lr-rule} is an equality of sets which can be proved by
\emph{double inclusion}. So that all the counting question is handled by the passage
to commutative image.

Here are the \Coq statements corresponding to this proof scheme. Note that to
be able to use finite sets, we have to confine the computation in a finite
type. We therefore fix the size \verb{d1} and \verb{d2} of the tableaux and
use the dependant types \verb{intpartn} and \verb{stdtabn}.

So here is the outline of the algebraic part of the proof:
\begin{itemize}
\item commutative image sends concatenation to ordinary product:
\begin{lstlisting}
Definition |*polyset*| d (s : {set d.-tuple 'I_n}) := \sum_(w in s) commword w.
Definition |*catset*| d1 d2 (s1 : {set d1.-tuple 'I_n}) (s2 : {set d2.-tuple 'I_n})
  :  {set (d1 + d2).-tuple 'I_n}
  := [set cat_tuple w1 w2 |  w1 in s1, w2 in s2].
Lemma |*multcatset*| d1 d2 (s1 : {set d1.-tuple 'I_n}) (s2 : {set d2.-tuple 'I_n}) 
  : polyset s1 * polyset s2 = polyset (catset s1 s2).
\end{lstlisting}
\item free Schur function and their commutative image:
\begin{lstlisting}
Definition |*freeSchur*| (Q : stdtabn d) :=
  [set t : d.-tuple 'I_n | (RStabmap t).2 == Q].
(* Size dependant version of the shape *)
Lemma |*is_part_shape_deg*| (Q : stdtabn d) : is_part_of_n d (shape Q).
Definition |*shape_deg*| (Q : stdtabn d) := (IntPartN (is_part_shape_deg Q)).

Lemma |*Schur_freeSchurE*| d (Q : stdtabn d) :
   Schur (shape_deg Q) = polyset R (freeSchur Q).
\end{lstlisting}
\item and the free \LR rule and its commutative image:
\begin{lstlisting}
Definition |*LR_support*| :=
  [set Q : stdtabn (d1 + d2) | predLRTripleFast Q1 Q2 Q ].
Lemma |*free_LR_rule*| :
  catset (freeSchur Q1) (freeSchur Q2) = \bigcup_(Q in LR_support) freeSchur Q.
Theorem |*LR_rule_tab*| :
  Schur (shape_deg Q1) * Schur (shape_deg Q2) =
    \sum_(Q in LR_support) Schur (shape_deg Q).
\end{lstlisting}
\item choosing a particular tableau called hyper-standard for each shape
  one gets a first statement for the product of Schur function:
\begin{lstlisting}
Definition |*LRtab_set*| (P : intpartn (d1 + d2)) :=
  [set Q in (LR_support (hyper_stdtab P1) (hyper_stdtab P2)) | (shape Q == P)].
Definition |*LRtab_coeff*| (P : intpartn (d1 + d2)) := #|LRtab_set P|.

Theorem |*LRtab_coeffP*| :
  Schur P1 * Schur P2 = \sum_P (Schur P) *+ LRtab_coeff P.
\end{lstlisting}
\item Once there, the remaining of the proof consist in constructing
  a bijection \verb{bijLR} between the Yamanouchi tableau of
  Theorem~\ref{theo:LR-rule} and the sets $\{t\in\operatorname{LR}(c, d) \mid
  \shape(t) = \nu\}$ (\verb{LRtab_set} in \Coq).
\begin{lstlisting}
Definition |*LRyam_set*| :=
  [set y : yameval_finType (intpartnP P2) | is_skew_reshape_tableau P P1 y].
Definition |*bijLR*| (yam : yameval P2) : stdtabn (d1 + d2) := [...]
Lemma |*bijLR_inj*| : {in LRyam_set &, injective bijLR}.
Lemma |*bijLR_image*| : LRtab_set P1 P2 P = [set bijLR x | x in LRyam_set].
Theorem |*LR_coeff_yamP*| : LRtab_coeff P1 P2 P = LRyam_coeff.
\end{lstlisting}
\end{itemize}
\bigskip

So going back to Theorem~\ref{theo:free-LR_rule} and the definition of free
Schur function, the crucial ingredient is to define the map $Q$ and the image
of the concatenation of two words under this map. This maps records what
happens during the execution of an algorithms dues to Schensted's. So that the
essential combinatorial part of the proof is an in depth study of this
algorithm.

\section{Schensted algorithm and the Robinson-Schensted bijection}

\begin{DEFN}
  Let $w = w_0\dots w_{l-1}$ a finite sequence over a totally ordered set. A
  \emph{subsequence} of $w$ is a sequence $v=v_0\dots v_{r-1}$ such that there exists
  a strictly increasing $r$-tuple of integer $I:=0<=i_0<i_1<\dots<i_{r-1}<l$
  verifying for all $k<r$, $v_k = w_{i_k}$.
\end{DEFN}
It is already defined in \SSR by the following boolean predicate:

\begin{lstlisting}
Fixpoint subseq s1 s2 :=
  if s2 is y :: s2' then
    if s1 is x :: s1' then subseq (if x == y then s1' else s1) s2' else true
  else s1 == [::].
\end{lstlisting}


The main algorithmic ingredient of the proof is Schensted algorithms which
solve efficiently the following problem
\begin{PROBLEM}
  Given a finite sequence $w$ over a totally ordered set, compute the maximum
  length of a increasing subsequence.
\end{PROBLEM}

  \begin{ALGO}
    Start with an empty row $r$, insert the letters $l$ of the word one by one
    from left to right by the following rule:
    \begin{itemize}
    \item replace the first letter strictly larger that $l$ by $l$;
    \item append $l$ to $r$ if there is no such letter.
    \end{itemize}
  \end{ALGO}
  \bigskip

\newcommand{\ar}[1]{\xrightarrow{#1}}
  Insertion of $\begin{array}{c@{}c@{}c@{}c@{}c@{}c@{}c@{}c@{}c@{}c@{}c@{}c@{}c}
    a&b&a&b&c&a&b&b&a&d&b&a&b\\
  \end{array}$
  \begin{multline*}
  \emptyset\ar{a}\young(a)\ar{b}\young(ab)\ar{a}\young(aa)\ar{b}\young(aab)
  \ar{c}\young(aabc)\ar{a}\young(aaac)\ar{b}\young(aaab)\ar{b}\\
  \young(aaabb)\ar{a}\young(aaaab)\ar{d}
  \young(aaaabd)\ar{b}\young(aaaabb)\ar{a}\\
  \young(aaaaab)\ar{b}\young(aaaaabb)
  \end{multline*}

\section{The plactic monoïd}

\section{Green's plactic invariants}

\section{Standardization}

\section{Shuffle product and the free \LR rule}

\section{The plactic version of the \LR rule}

\section{The final bijection}

\section{A \Coq implementation of the rule}
\label{implem}

\section{Concluding remarks}

\end{document}

%%% Local Variables:
%%% compile-command: "pdflatex -shell-escape lrproof.tex"
%%% mode: latex
%%% TeX-master: t
%%% End:
