\documentclass[12pt,a4paper]{article}

\usepackage[top=30pt,bottom=30pt,left=48pt,right=46pt]{geometry}

\usepackage{genyoungtabtikz}
% \YFrench % use french convention for tableaux.

\usepackage{minted}
\usemintedstyle{emacs}
%\usemintedstyle{colorful}
%\usemintedstyle{borland}
%\usemintedstyle{autumn}

\newminted{coq}{
frame=lines,
framesep=2mm,
fontsize=\scriptsize,
mathescape=true
}
\usepackage{commath}

% INFO DOCUMENT - TITRE, AUTEUR, INSTITUTION
\title{\bf\LARGE A formal proof of \\
Littlewood-Richardson rule\\[5mm]}
\author{Florent Hivert}
%\institute[LRI]{
%  LRI / Université Paris Sud 11 / CNRS / INRIA}
\date{Mai 2015}

\newcommand{\XX}{{\mathbb X}}

\newcommand{\exec}{\operatorname{exec}}
\newcommand{\free}[1]{\left\langle#1\right\rangle}
\newcommand{\gl}{{\mathfrak gl}}
\newcommand{\bb}{\mathbb}
\newcommand{\A}{{\mathbb A}}
\newcommand{\E}{{\mathbb E}}
\newcommand{\F}{\mathbf{F}}
\newcommand{\GG}{\mathbf{G}}
\newcommand{\G}{\mathbf{G}}
\newcommand{\N}{{\mathbb N}}
\newcommand{\C}{{\mathbb C}}
\newcommand{\R}{{\mathbb R}}
\newcommand{\Z}{{\mathbb Z}}
\newcommand{\Q}{{\mathbb Q}}
\newcommand{\SG}{{\mathfrak S}}
\newcommand{\ft}{\tilde{f}}
\newcommand{\et}{\tilde{e}}
\newcommand{\std}{\operatorname{Std}}
\newcommand{\Des}{\operatorname{Des}}
\newcommand{\Rec}{\operatorname{Rec}}
\newcommand{\Sh}{\operatorname{Sh}}

\newcommand{\sym}{\mathrm{sym}}
\newcommand{\NCSF}{\mathbf{NCSF}}
\newcommand{\QSym}{\mathrm{QSym}}
\newcommand{\FSym}{\mathbf{FSym}}
\newcommand{\FQSym}{\mathbf{FQSym}}
\newcommand{\MQSym}{\mathbf{MQSym}}
\newcommand{\Tree}{\mathbf{Tree}}

\newcommand{\MS}{\mathbf{MS}}
\newcommand{\FS}{{\mathbb \phi S}}
\newcommand{\SSs}{\mathbb{S}}
\newcommand{\TT}{{\cal T}}
\newcommand{\PP}{{\cal P}}
\def\Pp{{\bf P}}        % P de PBT
\def\Qq{{\bf Q}}        % Q de PBT^*
\newcommand{\HQ}{{\sf Q}}
\newcommand{\HR}{{\sf R}}
\newcommand{\rad}{\operatorname{rad}}
\newcommand{\soc}{\operatorname{soc}}
\newcommand{\Ext}{\operatorname{Ext}}
\newcommand{\End}{\operatorname{End}}
\newcommand{\conn}{{\cal C}}
\newcommand{\ff}{{\sf F}}
\newcommand{\UU}{\mathbf{U}}
\newcommand{\VV}{\mathbf{V}}
\newcommand{\sfact}{\operatorname{\rm sfact}}

\newcommand{\partof}{\vdash}                    % Partition de
\newcommand{\compof}{\vDash}                    % Composisition de
\newcommand{\ssh}{\Cup}
\newcommand{\saug}{\uplus}
\newcommand{\sconc}{\bullet}
\newcommand{\Std}{{\rm Std}}
\newcommand{\Park}{{\rm Park}}
\newcommand{\sumord}{\hat +}

\newcommand{\qandq}{\text{\quad et\quad}}

\newcommand{\tensor}{\otimes}
\newcommand{\pairing}[2]{\left\langle#1,#2\right\rangle} % Crochet de dualite

\newcommand{\red}[1]{{\color{red} #1}}
\newcommand{\grn}[1]{{\color{green} #1}}
\newcommand{\blu}[1]{{\color{blue} #1}}

%%%%%%%%%%%%%%%%%%%%%
\newcommand{\alphX}{{\mathbb X}}
\renewcommand{\emph}[1]{{\color{red} #1}}


\newtheorem{THEO}{Theorem}
\newtheorem{PROP}{Proposition}
\newtheorem{LEMMA}{Lemma}
\newtheorem{CORO}{Corollary}
\newtheorem{PROBLEM}{Problem}
\newtheorem{REMARK}{Remark}
\newtheorem{NOTE}{Note}

% \theoremstyle{definition}
\newtheorem{DEFN}{Definition}
\newtheorem{DEFNs}{Definitions}
\newtheorem{ALGO}{Algorithm}


%------------------------------------------------------------------------------
\begin{document}

\maketitle

\section{The rule}

\begin{THEO}[Littlewood-Richardson rule]
  $c_{\lambda, \mu}^{\nu}$ is the number of (skew) tableaux of shape the
  difference $\nu/\lambda$, whose row reading is a Yamanouchi word of
  evaluation $\mu$.
\end{THEO}

% ...00 ...00 ...00
% ...1  ...1  ...1 
% .00   .01   .02  
% 12    02    01   

Some examples:
  \[
  C_{331,421}^{5432} = 3
  \qquad
  \Yboxdim{12pt}\scriptstyle
  \gyoung(12,:;00,:::;1,:::;00)\qquad
  \gyoung(02,:;01,:::;1,:::;00)\qquad
  \gyoung(01,:;02,:::;1,:::;00)
  \]
% ...00 ...00 ...00 ...00 ...00 ...00 ...00 ...00 ...00 ...00 ...00 ...00 ...00 ...00 ...00
% ...1  ...1  ...1  ...1  ...1  ...1  ...1  ...1  ...1  ...1  ...1  ...1  ...1  ...1  ...1
% .00   .00   .00   .01   .00   .01   .02   .02   .12   .02   .02   .01   .01   .12   .12
% 01    02    11    01    12    02    01    13    03    03    11    12    22    02    23

  \[
  C_{4321,431}^{7542} = 4
  \qquad
  \Yboxdim{12pt}\scriptstyle
  \gyoung(:;2,::;11,:::;01,::::;000)\quad
  \gyoung(:;1,::;12,:::;01,::::;000)\quad
  \gyoung(:;1,::;02,:::;11,::::;000)\quad
  \gyoung(:;0,::;12,:::;11,::::;000)\quad
  \]


  \def\AA{\red 0}
  \def\AB{\grn 1}
  \def\AC{\blu 2}
  \def\AD{{\color{pink} 3}}
  \[
  C_{431,4321}^{7542} = 4
  \qquad
  \Yboxdim{12pt}\scriptstyle
  \gyoung(\AC\AD,:;\AB\AB\AC,:::;\AA\AB,::::;\AA\AA\AA)\quad
  \gyoung(\AC\AD,:;\AA\AB\AC,:::;\AB\AB,::::;\AA\AA\AA)\quad
  \gyoung(\AB\AD,:;\AA\AC\AC,:::;\AB\AB,::::;\AA\AA\AA)\quad
  \gyoung(\AA\AD,:;\AB\AC\AC,:::;\AB\AB,::::;\AA\AA\AA)\quad
  \]

  \def\AA{\red 0}
  \def\AB{\grn 1}
  \def\AC{\blu 2}
  \def\AD{{\color{pink} 3}}
  \[
  C_{431,4322}^{7543} = 2
  \qquad
  \gyoung(\AB\AD\AD,:;\AA\AC\AC,:::;\AB\AB,::::;\AA\AA\AA)\quad
  \gyoung(\AA\AD\AD,:;\AB\AC\AC,:::;\AB\AB,::::;\AA\AA\AA)\quad
  \]


\end{document}

%%% Local Variables: compile-command: "pdflatex -shell-escape lrproof.tex"
%%% Local Variables: LaTeX-command
%%% mode: latex
%%% TeX-master: t
%%% End:
